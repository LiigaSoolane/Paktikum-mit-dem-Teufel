\section{Theorie}
\label{sec:Theorie}

Temperatur innerhalb eines Systems verändert sich lokal durch Konvektion, Wärmestrahlung
und Wärmeleitung. Im vorliegenden Versuch wird die Wärmeleitung von Metallen untersucht;
dort überwiegt die Wärmeleitung eindeutig. Deshalb vernachlässigen wir im Folgenden die 
anderen beiden Optionen der Temperaturübertragung.\\
Für die Wärmemenge, die durch Wärmeleitung auf einer Oberfläche A pro Zeit übertragen 
wird, gilt
\begin{equation}
    \dfrac{dQ}{dt}=-\kappa A \dfrac{\partial T}{\partial x}
\end{equation}
wobei $\kappa$ die Wärmeleitfähigkeit des Materials ist. Für die Wärmestromdichte gilt 
daher:
\begin{equation}
    j_w=-\kappa \dfrac{\partial T}{\partial x}
\end{equation}
Unter Verwendung der Kontinuitätsgleichung
\begin{equation}
    \dfrac{\partial T}{\partial t}-\dfrac{\partial j}{\partial x}=0
\end{equation}
kann dieses Problem auf die eindimensionale Wärmeleitungs
\begin{equation}
    \dfrac{\partial T}{\partial t}=\dfrac{\kappa}{\rho c}\dfrac{\partial^2 T}{\partial x^2}
\end{equation}
zurückgeführt werden. $\dfrac{\kappa}{\rho c} = \sigma_T$ wird als Temperaturleitfähigkeit
bezeichnet. Sie ist materialabhängig und ist ein Kriterium für die jeweilige Geschwindigkeit 
des Temperaturausgleiches.\\
Im hier vorliegenden Fall eines langen, dünnen Stabes, der an einer Seite (x=0) abwechselnd 
erhitzt und gekühlt wird, ist 
\begin{equation}
    T(x,t)=T_{max} \exp{-\sqrt{\dfrac{\omega \rho c}{2 \kappa}}} \cos{\omega t - \sqrt{
        \dfrac{\omega \rho c}{2 \kappa}}}
\end{equation}
eine Lösung der Differentialgleichung. Die Phasengeschwindigkeit der Ausbreitung lautet also
\begin{equation}
    v = \dfrac{\omega}{k}=\dfrac{\omega}{\sqrt{\dfrac{\omega \rho c}{2 \kappa}}}
      = \sqrt{\dfrac{2 \kappa \omega}{\rho c}}
\end{equation}
und die Wärmeleitfähigkeit
\begin{equation}
    \kappa = \dfrac{\rho c (\Delta x)^2}{2 \Delta t \ln{A_{nah}}{A_{fern}}}.
\end{equation}
Dabei ist $\Delta x$ der Abstand zwischen den beiden Messpunkten, $\Delta t$ die Phasendifferenz,
und $A_{nah}$ bzw. $A_{fern}$ die jeweilige Amplitude.