\section{Durchführung}
\label{sec:Durchführung}
An einer strombetriebenen Wärmequelle sind einseitig vier Metallstäbe aus verschiedenen 
Materialien, Edelstahl, Aluminium, ein schmaler Messingstab und ein breiter Messingstab, 
angebracht. An jeweils zwei, voneinander 3 cm entfernten Stellen der vier Metallproben sind 
Temperaturmessgeräte installiert. Das Heizelement wird über einen Schalter auf \textit{Heat}
oder \textit{Cool} geregelt. Die Daten Temperaturmessgeräte werden
mithilfe eines Datenloggers aufgenommen, wobei eine Abstastrate von $\Delta s= 10$ s verwendet wird.


\subsection{Statische Methode}
    In diesem Versuchteil wird das Heizelement
    durch den konstanten Strom von 5 V beheizt. Der Temperaturverlauf
    wird über ein Zeitintervall von 1000 s 
    verfolgt. Die Daten werden mit einem Datenlogger simultan aufgenommen. 
    Durch den aufgenommenen Temperaturverlauf wird die Wärmeleitfähigkeit bestimmt.

\subsection{Dynamische Methode}
    Als dynamische Methode wird in diesem Versuch das Angström-Messverfahren verwendet. Dabei werden
    die zu untersuchende Stab über eine Periode von 80 s abwechselnd geheizt und gekühlt.
    Die Wärmequelle wird nun über eine Spannung von 8 V betrieben.\\
    Die erste Messreihe dieser Art wird mit einer Periodendauer von 80 s über 
    10 Perioden durchgeführt.\\
    In einer weiteren Messreihe wird eine Periodendauer von 200 s über 6 Perioden verwendet.


    \label{sec:aufbau}
    Der Versuchsaufbau besteht aus einem PCB, auf welchem acht Thermoelemente $T1$ bis
    $T8$ an drei Materialien Aluminium, Edelstahl und Messing angebracht sind. Die
    Materialien werden von einem Peltierelement geheizt beziehungsweise gekühlt.
    Der Betriebsmodus des Peltierelements kann über einen Schalter auf \textit{Heat}
    oder \textit{Cool} geregelt werden. An das Peltierelement ist für die statische
    Methode eine Spannung $U_P = 5 \si{\volt}$, für die dynamische Methode eine
    Spannung von $U_P = 8 \si{\volt}$ angelegt. Die Daten der Thermoelemente werden
    mithilfe eines Datenloggers \textit{Xplorer GLX} aufgenommen.
    Der Versuchsaufbau ist in \ref{fig:aufbau} zu erkennen.
    
    \begin{figure}[H]
      \centering
      \includegraphics[scale=0.5]{content/pcb.png}
      \caption{Foto des Versuchsaufbaus.\cite{AP01}}
      \label{fig:aufbau}
    \end{figure}
    
    \subsection{Die statische Methode}
    \label{sec:statisch}
    Hier wird an zwei Punkten jeder Metallprobe die Temperatur als Funktion der Zeit
    gemessen. Durch den aufgenommenen Temperaturverlauf wird die Wärmeleitfähigkeit bestimmt.
    Der Datenlogger wird auf eine Abtastrate von $\Delta t_{GLX} = 10 \si{\second}$
    gestellt. Die Sensoren werden im \textit{Home}-Verzeichnis unter dem Reiter \textit{Digital}
    angezeigt. Das Netzteil wird auf eine Spannung von $U_P = 5 \si{\volt}$ eingestellt.
    Die Messung wird beendet, wenn das Thermoelement $T7$ eine Temperatur von $45 \si{\celsius}$
    anzeigt.
    
    \subsection{Die dynamische Methode}
    \label{sec:dynamisch}
    Bei der dynamischen Methode oder dem Angström-Messverfahren wird der Probenstab
    periodisch geheizt. Aus der Ausbreitungsgeschwindigkeit der Temperaturwelle lässt sich
    dann die Wärmeleitfähigkeit bestimmen. Die Sensoren können wieder im Reiter \textit{Digital}
    des Datenloggers eingesehen werden. Der Datenlogger wird auf eine Abtastrate von
    $\Delta t_{GLX} = 2 \si{\second}$ eingestellt. Es wird eine Messung durchgeführt,
    bei der die Probenelemente mit $80 \si{\second}$-Perioden geheizt und gekühlt werden.
    Nun wird das Netzteil auf $U_P = 8 \si{\volt}$ und maximalen Strom eingestellt.
    Die Messung wird erneut durchgeführt und erst bei mindestens $10$ Perioden eingestellt.
    Eine dritte Messung wird mit einer Periodendauer von $200 \si{\second}$ durchgeführt.
    Diese wird beendet, wenn eines der Probenelemente eine Temperatur von $80 \si{\celsius}$
    erreicht. Anschließend werden die Elemente abgekühlt.