\section{Auswertung}
\label{sec:Auswertung}

\subsection{Messung des Magnetfeldes}
Zunächst wurde mithilfe einer kleinen Hall-Sonde das von den Spulen erzeugte Magnetfeld gemessen, dazu wurde der Strom, der in die Spulen geleitet wurde, stückweise erhöht, 
nach Umpolung der Stromzufuhr wurde die Messung wiederholt. Es ergaben sich die in Tabelle \ref{tab:mag}.
%\begin{tabular}
% \centering
% \label{tab:mag}
% \caption{Messung des Magnetfeldes.}
% 
%\end{tabular}
Beide Verläufe wurden mithilfe linearer Regression in kontinuierliche Verläufe der Form $y = a + b*x$ überführt, durch Rechnung mit Scientific Python ergeben sich für das 
Magnetfeld vor der Umpolung
\begin{align*}
a = 0.204 \pm 0.002
b = 0.011 \pm 0.005
B = (0.204 \pm 0.002) + (0.011 \pm 0.005) \cdot I
\end{align*}
und für das Magnetfeld nach Umpolung
\begin{align*}
a = 0.19 \pm 0.005
b = 0.02 \pm 0.014
B = (0.19 \pm 0.005) + (0.02 \pm 0.014) \cdot I
\end{align*}
Die Abbildung \ref{fig:plots} zeigt das jeweilige Magnetfeld aufgetragen gegen die Stromstärke. 
%\begin{figure}
% \centering
% \includegraphics[width=\textwidth]{}
Um genauere Werte zum Weiterrechnen zu erzielen, wurde der Mittelwert aus
beiden Geraden verwendet. Dann folgt eine Relation von Magnetfeld und Stromstärke nach
%\begin{align*}
%
%\end{align*}

\subsection{Berechnung der Hall-Spannung}

\subsection{Berechnung der mikroskopischen Leitfähigkeitsparameter}