\section{Durchführung}
\label{sec:Durchfuehrung}

Zunächst wird der Elektrische Widerstand, die Länge und der Querschnitt eines Kupferdrahtes bestimmt,
um daraus den speziefischen Widerstand von Kupfer ableiten zu können. 
Vorbereitend wird außerdem das durch Elektromagneten erzeugte Magnetfeld 
unter Variierung des Stroms gemessen, um statistischen Messunsicherheiten vorzubeugen.
Dann wird eine zweite, quaderförmige Probe als Hallsonde verwendet. Diese wird ebenfalls
abgemessen. 
Damit sind die Vorbereitungen abgeschlossen und die eigentliche Messung beginnt.
Der Versuchsaufbau wurde bereits in der Graphik \ref{fig:hall} dargestellt.
Durch Elektromagneten wird ein elektrisches Feld erzeugt, in dem die Folie eines Materials,
(hier Kupfer) senkrecht zur Magnetfeldausrichtung liegt. Es wird die Spannung zwischen 
den Punkten A und B abggegriffen, die sog. Hallspannung.
Es werden zwei Messreihen durchgeführt. Einmal wird erneut die Stromstärke der 
Elektromagneten variiert. Bei der zweiten Messreihe wird die Stromstärke durch die Hallsonde variiert.
