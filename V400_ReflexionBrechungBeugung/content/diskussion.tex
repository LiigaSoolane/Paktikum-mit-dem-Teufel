\section{Diskussion}
\label{sec:Diskussion}

Die gemessenen Werte lauten


Während die Theoriewerte

sind. 

%noch prozentuale Abweichungen bestimmen?

Die Abweichungen haben größtenteils systematische Ursachen. Da die Werte stets analog anhand einer Unterlage abgemessen wurden, sind Fehler beim Ablesen sehr wahrscheinlich.
Da die Richtigkeit der abgelesenen Werte auch davon abhing, ob die Versuchsanordnung korrekt auf die Unterlage ausgerichtet war, kommen als weitere Fehlerquellen ein Verrutschen
der Unterlage oder der Versuchsanordnung hinzu, die zu Abweichungen beitragen.\\
Hinzu kommt außerdem bei der reflektierenden Platte, dass sie mit großer Wahrscheinlichkeit nicht perfekt glatt war. Dadurch wird das Licht nicht einwandfrei zurückgeworfen
und es kommt zu Abweichungen im Ausfallswinkel. \\
Bei der Planparallelen Platte können mögliche Verunreinigungen des Plexiglases beispielsweise durch Fingerabdrücke oder Kratzer für Fehler im Brechungswinkel oder 
Strahlversatz führen. \\
Beim Versuchsaufbau für die Untersuchung der Beugung am Gitter ist zu bedenken, dass das Papier mit Winkelskala von Hand aufgestellt wurde, also keinen perfekten Bogen beschrieben 
hat. So kommen auch hier systematische Messunsicherheiten zustande.\\
Durch Anwendung der Gauss'schen Fehlerfortpflanzung in den Rechnungen verstärken sich die Fehler gegenständig. Im allgemeinen ist aber kein Wert gravierend vom jeweiligen 
Theoriewert abgewichen.