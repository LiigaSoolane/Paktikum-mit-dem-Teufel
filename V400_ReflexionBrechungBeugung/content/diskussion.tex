\section{Diskussion}
\label{sec:Diskussion}

Wie zu erwarten, liegt die Steigung der linearen Regression der Ausfallswinkel in Aufgabenteil 1 bei 
$1.001 \pm 0.007$. Die Abweichung vom Erwartungswert 1 liegen somit noch innerhalb der Fehlergrenzen.
Die Abweichungen haben größtenteils systematische Ursachen. Da die Werte stets analog 
anhand einer Unterlage abgemessen wurden, sind Fehler von bis zu einem Grad beim Ablesen sehr wahrscheinlich.
Nicht zuletzt, weil man durch ein Glas hindurch auf die Unterlage schaut, so dass auch beim Ablesen
der Brechungseffekt sehr gut erkennbar ist. Hinzu kommt außerdem bei der reflektierenden 
Platte, dass sie mit großer Wahrscheinlichkeit nicht perfekt glatt war. Dadurch wird das 
Licht nicht einwandfrei zurückgeworfen und es kommt zu Abweichungen im Ausfallswinkel. \\
 
\noindent Kronglas besitzt nach \cite{brechind} einen Brechungsindex von 1.5-1.6, sodass der hier bestimmte Wert von 
1.53 mit den Theoriewerten übereinstimmt. Bei der Planparallelen Platte können mögliche Verunreinigungen des 
Plexiglases beispielsweise durch Fingerabdrücke oder Kratzer für Fehler im Brechungswinkel oder 
Strahlversatz führen. \\

\noindent Die vom Hersteller angegebene Wellenlänge liegt mit 635 nm deutlich unter der durch den Versuch 
bestimmten von 650 nm. Zwischen den einzelnen Messpunkten liegen keine großen Abweichungen 
vor, so dass es sich vermutlich um statistische Messfehler handelt. Beim Versuchsaufbau ist 
zu bedenken, dass das Papier mit Winkelskala von Hand aufgestellt wurde, also keinen perfekten 
Bogen beschrieb. Da die Richtigkeit der abgelesenen Werte auch davon abhing, ob die Versuchsanordnung
korrekt auf die Unterlage ausgerichtet war, kommen als weitere Fehlerquellen ein Verrutschen
der Unterlage oder der Versuchsanordnung hinzu, die zu Abweichungen beizutragen.\\
%noch prozentuale Abweichungen bestimmen?
