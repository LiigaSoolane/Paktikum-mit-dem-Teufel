\section{Diskussion}
\label{sec:Diskussion}
\subsection{Erste Versuchsreihe: Niedrigdruckbereich}
Die Messwerte lauten:
\begin{table}[H]
\centering
   \caption{Messwerte bis 1 bar}
   \label{tab:ndr}
   \begin{tabular}{c c}
   \toprule
    $p$[Pa] & $T$[K] \\
    \midrule
      3700 &   301.65 \\  
      5700 &   309.15 \\  
      7700 &   314.15 \\  
      9700 &   318.15 \\ 
      11700 &   322.15 \\ 
      13700 &   322.15 \\ 
      15700 &   328.15 \\ 
      17700 &   330.15 \\ 
      19700 &   333.15 \\ 
      21700 &   335.15 \\ 
      23700 &   337.15 \\ 
      35700 &   339.15 \\ 
      27700 &   341.15 \\ 
      29700 &   342.15 \\ 
      31700 &   344.15 \\ 
      33700 &   346.15 \\ 
      35700 &   347.15 \\ 
      37700 &   349.15 \\
      39700 &   350.15 \\ 
      41700 &   352.15 \\ 
      43700 &   353.15 \\ 
      45700 &   355.15 \\ 
      47700 &   356.15 \\ 
      49700 &   357.15 \\ 
      51700 &   358.15 \\ 
      53700 &   359.15 \\ 
      55700 &   360.15 \\ 
      57700 &   361.15 \\ 
      59700 &   362.15 \\ 
      61700 &   363.15 \\ 
      63700 &   364.15 \\ 
      65700 &   365.15 \\ 
      67700 &   366.15 \\ 
      69700 &   367.15 \\ 
      71700 &   368.15 \\ 
      73700 &   369.15 \\ 
      75700 &   369.15 \\ 
      77700 &   370.15 \\ 
      79700 &   371.15 \\ 
      81700 &   372.15 \\ 
      83700 &   372.15 \\ 
      85700 &   373.15 \\ 
      87700 &   373.15 \\
      89700 &   375.15 \\ 
      91700 &   375.15 \\ 
      93700 &   376.15 \\ 
      95700 &   377.15 \\ 
      97700 &   377.15 \\ 
      99700 &   378.15 \\ 
      101700 &   378.15 \\ 
    \bottomrule
    \end{tabular}
\end{table}
Aus Ihnen ergibt sich die Verdampfungwärme L als 
\begin{equation*}
    L= (3.9 \pm 0.05) \cdot 10^4 \dfrac{J}{mol}
\end{equation*}
In der Quelle \cite{Verdampfungwärme} wurden die in Tabelle \ref{tab:ndrtheo} dargestellten Werte
für die Verdampfungswärme festgestellt. Beachtet werden dabei nur Werte, die im gleichen 
Druck- und Temperaturbereich liegen wie die gemessenen.
\begin{table}[H]
\centering
   \caption{Messwerte des Hochdruckbereichs}
   \label{tab:ndrtheo}
   \begin{tabular}{c c}
   \toprule
    $T$[K] & $L$[J/mol] \\
    \midrule
    298.15 &    43990 \\ 
    313.15 &    43350 \\   
    333.15 &    42483 \\   
    353.15 &    41585 \\   
    373.15 &    40657 \\   
    \bottomrule
    \end{tabular}
\end{table}
Diese liegen außerhalb der Fehlergrenze des in der Auswertung berechneten Wertes für die 
Verdampfungswärme. Allerdings weicht die berechnete Verdampfungswärme nur um  stimmt aber überein und die Abweichung von den 
Literaturwerten ist vermutlich durch Fehler während den Messungen entstanden. ???What
Zunächst wurde es versäumt, die Apparatur einzuschalten, was zu einem langsamen Temperaturanstieg über einen längeren Zeitraum geführt hat. Als der Fehler aufgefallen
ist musste gewartet werden, bis die Substanz wieder auf Zimmertemperatur abkühlte. Aus Zeitgründen konnte das nicht gänzlich realisiert werden, weshalb die Starttemperatur 
mit $ 28.5 \si{degreeCelsius}$ deutlich höher ausfiel als in der Versuchsanleitung beabsichtigt. Es ist zu erwarten, dass sich so ein negativer Einfluss auf die restliche Messung
ergeben hat. \\
Auch der Fehler, der beim Ablesen von Thermo- oder Manometer entsteht, ist erwähnenswert. Da das Manometer ein Digitales war, ist der erwartete Messfehler für den ablegesenen 
Druck gering, allerdings wurde ein analoges Thermometer eingesetzt, was unweigerlich einen Messfehler mit sich bringt.
%Werte, die eine Sonderbehandlung brauchen?

\subsection{Zweite Versuchsreihe: Hochdruckbereich}
In Tabellle \ref{tab:hdr} sind die aus den Messwerten berechneten Werte für L, sowie Theoriewerte, die der
Quelle \cite{Dampfdrucktabelle} entnommen wurden und die prozentuale Abweichung dargestellt.
\begin{table}[H]
\centering
   \caption{}
   \label{tab:hdr}
   \begin{tabular}{c c c c}
   \toprule
     $p$[kPa] & $L_{theo}$[J/mol] & $L_{theo}$[J/mol] & Abweichung [\%] \\
    \midrule
      0.5000 & 41490.0000 & 50154.0020 &    17.2748 \\     1.0000 & 40644.0000 & 76271.3535 &    46.7113 \\     1.5000 & 40068.0000 & 72648.6618 &    44.8469 \\     2.0000 & 39618.0000 & 65511.7168 &    39.5253 \\     2.5000 & 39258.0000 & 61396.5847 &    36.0583 \\     3.0000 & 38934.0000 & 56627.2147 &    31.2451 \\     3.5000 & 38646.0000 & 52363.0790 &    26.1961 \\     4.0000 & 38394.0000 & 50044.3491 &    23.2800 \\     4.5000 & 38160.0000 & 47893.1901 &    20.3227 \\     5.0000 & 37926.0000 & 45896.9634 &    17.3671 \\     5.5000 & 37728.0000 & 43746.4241 &    13.7575 \\     6.0000 & 37530.0000 & 42317.3771 &    11.3130 \\     - &     0.0000 & 41233.0450 &   - \\     7.0000 & 37170.0000 & 39207.5587 &     5.1969 \\     7.5000 &     - & 37801.9554 &   - \\     8.0000 & 36828.0000 & 36483.5203 &    -0.9442 \\ 
    \bottomrule
    \end{tabular}
\end{table}
Wie zu erkennen ist, gab es bei der Messung höchstwahrscheinlich gravierende Missstände.
Wie bereits beschrieben begann die Apparatur bei einem Druck von $\symup{p} = \si{\bar}$ zu zischen, was verbunden mit dem beobachteten Druckabfall den Schluss nahelegt, dass
Gas aus dem Versuchsaufbau entwich. Dadurch konnte die Messung nicht in Gänze durchegeführt werden und wurde nach etwa der Hälfte abgebrochen, wodurch ein Teil der für die 
vollständige Darstellung nötigen Werte fehlen. Zu Bedenken ist auch, dass die mangelhafte Dichtung des Gerätes auch die vorher gemessenen Werte negativ beeinflusst haben 
könnte, da kein abgeschlossenes System mehr vorlag und so auch kein reversibler Prozess, was allerdings eine Grundannahme in der theoretischen Herleitung der Gleichung war. \\
Hinzu kommt der typische menschliche Fehler, der aus dem Ablesen an analogen Thermometern und Manometern folgt. Es ist zudem zu bemerken, dass die Nadel des Manometers bei 
Erschütterung des Untergrunds sprunghaft um wenige Millibar anstieg, sodass die abgelesenen Werte möglicherweise keinen gleichmäßigen Abstand zueinander haben. \\
Des Weiteren sind in die Herleitung der Differentialgleichung, nach der die Kurve bestimmt wurde, viele Näherungen eingegangen, die eine Abweichung von tatsächlichen Werten
zur Folge haben.
%irgendwelche Werte die gesondert diskutiert werden sollten?