\section{Auswertung}
\label{sec:Auswertung}
\subsection{Niedrigdruckbereich}

  Bei den Messungen bis 1 bar wurde die in folgender Tabelle dargestellten Werte festgehalten. Sie
  wurden bereits in die SI Einheiten Kelvin und Pascal umgerechnet.
  \begin{table}[H]
    \centering
   \caption{Messwerte bis 1 bar}
   \label{tab:data}
   \begin{tabular}{c c c c c}
   \toprule
    $p$[Pa] & $T$[K] \\
    \midrule
      3700 &   301.65 \\  
      5700 &   309.15 \\  
      7700 &   314.15 \\  
      9700 &   318.15 \\ 
      11700 &   322.15 \\ 
      13700 &   322.15 \\ 
      15700 &   328.15 \\ 
      17700 &   330.15 \\ 
      19700 &   333.15 \\ 
      21700 &   335.15 \\ 
      23700 &   337.15 \\ 
      35700 &   339.15 \\ 
      27700 &   341.15 \\ 
      29700 &   342.15 \\ 
      31700 &   344.15 \\ 
      33700 &   346.15 \\ 
      35700 &   347.15 \\ 
      37700 &   349.15 \\
      39700 &   350.15 \\ 
      41700 &   352.15 \\ 
      43700 &   353.15 \\ 
      45700 &   355.15 \\ 
      47700 &   356.15 \\ 
      49700 &   357.15 \\ 
      51700 &   358.15 \\ 
      53700 &   359.15 \\ 
      55700 &   360.15 \\ 
      57700 &   361.15 \\ 
      59700 &   362.15 \\ 
      61700 &   363.15 \\ 
      63700 &   364.15 \\ 
      65700 &   365.15 \\ 
      67700 &   366.15 \\ 
      69700 &   367.15 \\ 
      71700 &   368.15 \\ 
      73700 &   369.15 \\ 
      75700 &   369.15 \\ 
      77700 &   370.15 \\ 
      79700 &   371.15 \\ 
      81700 &   372.15 \\ 
      83700 &   372.15 \\ 
      85700 &   373.15 \\ 
      87700 &   373.15 \\
      89700 &   375.15 \\ 
      91700 &   375.15 \\ 
      93700 &   376.15 \\ 
      95700 &   377.15 \\ 
      97700 &   377.15 \\ 
      99700 &   378.15 \\ 
      101700 &   378.15 \\ 
    \bottomrule
    \end{tabular}
  \end{table}

  \noindent Zur Berechnung von L wird die in der Theorie hergeleitete Formel 
  \begin{equation}
    \label{eq:L}
    \ln{(\dfrac{p}{p_0})}=-\dfrac{L}{RT}\ \Leftrightarrow \ L=-\ln{(\dfrac{p}{p_0})}RT
  \end{equation} 
  verwendet. Der Umgebungsdruck auf Meereshöhe beträgt etwa
  $p_0=1 bar$. Die allgemeine Gaskonstante lautet $R=8.31446261815324
  {\displaystyle \textstyle {\frac {\mathrm {kg\,m^{2}} }{\mathrm {s^{2}\,mol\,K} }}} $. Somit lässt 
  sich L mittels linearer Regression aus folgender Grafik berechnen: 
  \begin{figure}[H]
   \centering
   \includegraphics{plot.pdf}
   \caption{Messwerte und Ausgleichsgerade bis 1 bar}
   \label{fig:plot}
  \end{figure}
  \noindent Für die Ausgleichsgerade ergeben sich die Koeffizienten
  \begin{align*}
   a=(-4686.5514 \pm 58.9156)\ K\\
   b=10.1354 \pm 0.2268
  \end{align*}
  \noindent Einsetzen in die Formel \eqref{eq:L} liefert:

  \begin{equation*}
    L= - a \cdot R = (3.9 \pm 0.05) \cdot 10^4 \dfrac{J}{mol}
  \end{equation*}
\subsection{Hochdruckbereich}

