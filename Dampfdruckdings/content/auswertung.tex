\section{Auswertung}
\label{sec:Auswertung}
\subsection{Niedrigdruckbereich}

  Bei den Messungen bis 1 bar wurden die in Tabelle \ref{tab:data} dargestellten Werte festgehalten. Sie
  wurden bereits in die SI Einheiten Kelvin und Pascal umgerechnet.
  \begin{table}[H]
    \centering
   \caption{Messwerte bis 1 bar}
   \label{tab:data}
   \begin{tabular}{c c}
   \toprule
    $p$[Pa] & $T$[K] \\
    \midrule
      3700 &   301.65 \\  
      5700 &   309.15 \\  
      7700 &   314.15 \\  
      9700 &   318.15 \\ 
      11700 &   322.15 \\ 
      13700 &   322.15 \\ 
      15700 &   328.15 \\ 
      17700 &   330.15 \\ 
      19700 &   333.15 \\ 
      21700 &   335.15 \\ 
      23700 &   337.15 \\ 
      35700 &   339.15 \\ 
      27700 &   341.15 \\ 
      29700 &   342.15 \\ 
      31700 &   344.15 \\ 
      33700 &   346.15 \\ 
      35700 &   347.15 \\ 
      37700 &   349.15 \\
      39700 &   350.15 \\ 
      41700 &   352.15 \\ 
      43700 &   353.15 \\ 
      45700 &   355.15 \\ 
      47700 &   356.15 \\ 
      49700 &   357.15 \\ 
      51700 &   358.15 \\ 
      53700 &   359.15 \\ 
      55700 &   360.15 \\ 
      57700 &   361.15 \\ 
      59700 &   362.15 \\ 
      61700 &   363.15 \\ 
      63700 &   364.15 \\ 
      65700 &   365.15 \\ 
      67700 &   366.15 \\ 
      69700 &   367.15 \\ 
      71700 &   368.15 \\ 
      73700 &   369.15 \\ 
      75700 &   369.15 \\ 
      77700 &   370.15 \\ 
      79700 &   371.15 \\ 
      81700 &   372.15 \\ 
      83700 &   372.15 \\ 
      85700 &   373.15 \\ 
      87700 &   373.15 \\
      89700 &   375.15 \\ 
      91700 &   375.15 \\ 
      93700 &   376.15 \\ 
      95700 &   377.15 \\ 
      97700 &   377.15 \\ 
      99700 &   378.15 \\ 
      101700 &   378.15 \\ 
    \bottomrule
    \end{tabular}
  \end{table}

  \noindent Zur Berechnung von L wird die in der Theorie hergeleitete Formel 
  \begin{equation}
    \label{eq:L}
    \ln{(\dfrac{p}{p_0})}=-\dfrac{L}{RT}\ \Leftrightarrow \ L=-\ln{(\dfrac{p}{p_0})}RT
  \end{equation} 
  verwendet. Der Umgebungsdruck auf Meereshöhe beträgt etwa
  $p_0=1 bar$. Die allgemeine Gaskonstante lautet $R=8.31446261815324
  {\displaystyle \textstyle {\frac {\mathrm {kg\,m^{2}} }{\mathrm {s^{2}\,mol\,K} }}} $. Somit lässt 
  sich L mittels linearer Regression aus folgender Grafik berechnen: 
  \begin{figure}[H]
   \centering
   \includegraphics{plot.pdf}
   \caption{Messwerte und Ausgleichsgerade bis 1 bar}
   \label{fig:plot}
  \end{figure}
  \noindent Für die Ausgleichsgerade ergeben sich die Koeffizienten
  \begin{align*}
   a=(-4686.5514 \pm 58.9156)\ K\\
   b=10.1354 \pm 0.2268
  \end{align*}
  \noindent Einsetzen in die Formel \eqref{eq:L} liefert:

  \begin{equation*}
    L= - a \cdot R = (3.9 \pm 0.05) \cdot 10^4 \dfrac{J}{\symup{mol}}
  \end{equation*}
  Die Unsicherheit wurde mit Numerik Python berechnet.\\
  Die Verdamfpungswären setzt sich aus zwei Teilen zusammen: Der äußeren Verdampfungswärme $L_a$
  liefert die Energie, die benötigt, wird um das Volumen einer Flüssigkeit in das eines 
  Gases zu vergrößern. Sie lässt sich also durch die Volumenarbeit $W = pV$ ausdrücken. Da es sich um 
  den Niedrigdruckbereich handelt, kann mit der idealen Gasgleichung $pV = RT$ gearbeitet 
  werden, so dass sich für 
  \begin{equation*}
  \symup{L_a} = \symup{W} = pV = RT
  \end{equation*}
  ergibt. Die innere Verdampfungswärme $L_i$, die benötigt wird, um die molekularen Bindungskräfte
  zu überwinden kann daher aus der Gleichung
  \begin{equation*}
  \symup{L_i = L - L_a}
  \end{equation*}
  gewonnen werden.

\subsection{Hochdruckbereich}
  Tabelle \ref{tab:data2} zeigt die Messwerte im Bereich von 50 bis 800 kPa.
  \begin{table}[H]
    \centering
     \caption{Messwerte ab 1 bar}
     \label{tab:data2}
     \begin{tabular}{c c}
     \toprule
     $p$[kPa] & $T$[K] \\
      \midrule
       50 &   381.15 \\ 
      100 &   392.15 \\ 
      150 &   398.15 \\ 
      200 &   405.15 \\ 
      250 &   409.15 \\ 
      300 &   414.15 \\ 
      350 &   419.15 \\ 
      400 &   422.15 \\ 
      450 &   425.15 \\ 
      500 &   428.15 \\ 
      550 &   431.65 \\ 
      600 &   434.15 \\ 
      650 &   436.15 \\ 
      700 &   440.15 \\ 
      750 &   443.15 \\ 
      800 &   446.15 \\ 
     \bottomrule
    \end{tabular}
  \end{table}
  \begin{figure}[H]
   \centering
   \includegraphics{plot1.pdf}
   \caption{Messwerte im Hochdruckbereich}
   \label{fig:plot1}
  \end{figure}
  \noindent Wie in Abbildung \ref{fig:plot1} bereits dargestellt, können die Messwerte durch ein Polynom
  der Form $p = a \cdot T³ + b \cdot T² + c \cdot T + d$ genähert werden. Die mit Python
  ermittelten Koeffizienten lauten:
  \begin{align*}
    & a = -0.5015 \pm 0.3859 \ \dfrac{Pa}{K³}\\
    & b = 750.5024 \pm 479.7916 \ \dfrac{Pa}{K²}\\
    & c = (-3.5129 \pm 1.9861) \cdot 10⁵ \ \dfrac{Pa}{K}\\
    & d = (5.269 \pm 2.7371) \cdot 10⁷ \ Pa\\
  \end{align*}
  Die Verdampfungswärme L lässt sich bekanntermaßen aus der Clausius-Clapeyron Gleichung 
  \begin{equation}
    \label{eqn:ccg}
    (V_D-V_F)dp = \dfrac{L}{T}dT \Leftrightarrow L = (T \cdot \dfrac{dp}{dT}\cdot(V_D-V_F)) 
  \end{equation}
  berechnen.
  Die Ableitung des Drucks nach der Zeit beträgt offensichtlich $\dfrac{dp}{dT}= 3a \cdot
  T²+2b\cdot T+ c$. $V_F$ kann hier erneut vernachlässigt werden.
  Da hier zu hoher Druck herrscht, um von einem idealen Gas ausgehen zu können, wird 
  $V_D$ anhand der Gleichung
  \begin{equation*}
  (p+\dfrac{a}{V²})V = RT \Leftrightarrow V = \dfrac{RT}{2p} \pm \sqrt{\dfrac{R²T²}{4p²}-\dfrac{a}{p}}
  \end{equation*}
  bestimmt. Eingesetzt in Gleichung \ref{eqn:ccg} ergibt sich für die Verdampfungswärme:
  \begin{equation*}
  L = T \cdot \dfrac{dp}{dT} \cdot (\dfrac{RT}{2p} \pm \sqrt{\dfrac{R²T²}{4p²}-\dfrac{a}{p}})
  \end{equation*}
  Aus dieser Gleichung ergeben sich zwei unterschiedliche Kurven für die Verdampungswärme, die
  Folgenden dargestellt sind. Abbilgung \ref{fig:plot2} behandelt den Fall, dass die Wurzel aufaddiert wird; 
  Abbildung \ref{fig:plot3} den Fall, dass die Wurzel subtrahiert wird. Physikalisch sinnvoll ist allerdings
  nur der Term mit aufaddierter Wurzel.
  \begin{figure}[H]
   \centering
   \includegraphics{plot2.pdf}
   \caption{Verdampfungswärme mit positiver Wurzel}
   \label{fig:plot2}
  \end{figure}
  
  \begin{figure}[H]
   \centering
   \includegraphics{plot3.pdf}
   \caption{Verdampfungswärme mit negativer Wurzel}
   \label{fig:plot3}
  \end{figure}

