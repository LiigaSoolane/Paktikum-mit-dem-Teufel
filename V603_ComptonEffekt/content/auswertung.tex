\section{Auswertung}
\label{sec:Auswertung}

\subsection{Emissionsspektrum der Cu-Röntgenröhre}

Zunächst wird das Emissionsspektrum der Cu-Röntgenröhre 
aufgenommen, um es genauer analysieren zu können. 
Auf die Auflistung der korrespondierenden Werte wird 
an dieser Stelle aufgrund der großen Anzahl
an Messwerten verzichtet.\\
%Die aufgenommenen Daten sind in Tabelle \ref{tab:emiss} zu finden.

%\begin{table}
%    \centering
%    \begin{tabular}{c c c c}
%        \toprule
%        $\theta$ [°] & $N$ [Imp/s] & $\theta$ [°] & $N$ [Imp/s] \\
%        \midrule
%        11.0	&	414.0 & 18.2	&	180.0\\
%        11.1	&	420.0 & 18.3	&	166.0\\
%        11.2	&	417.0 & 18.4	&	173.0\\
%        11.3	&	417.0 & 18.5	&	167.0\\
%        11.4	&	409.0 & 18.6	&	169.0\\
%        11.5	&	406.0 & 18.7	&	160.0\\
%        11.6	&	404.0 & 18.8	&	159.0\\
%        11.7	&	405.0 & 18.9	&	157.0\\
%        11.8	&	400.0 & 19.0	&	149.0\\
%        11.9	&	383.0 & 19.1	&	153.0\\
%        12.0	&	389.0 & 19.2	&	150.0\\
%        12.1	&	382.0 & 19.3	&	147.0\\
%        12.2	&	372.0 & 19.4	&	150.0\\
%        12.3	&	376.0 & 19.5	&	148.0\\
%        12.4	&	385.0 & 19.6	&	149.0\\
%        12.5	&	384.0 & 19.7	&	143.0\\
%        12.6	&	382.0 & 19.8	&	153.0\\
%        12.7	&	373.0 & 19.9	&	182.0\\
%        12.8	&	376.0 & 20.0	&	291.0\\
%        12.9	&	373.0 & 20.1	&	1127.0\\
%        13.0	&	375.0 & 20.2	&	1599.0\\
%        13.1	&	366.0 & 20.3	&	1533.0\\
%        13.2	&	354.0 & 20.4	&	1430.0\\
%        13.3	&	341.0 & 20.5	&	1267.0\\
%        13.4	&	326.0 & 20.6	&	425.0\\
%        13.5	&	318.0 & 20.7	&	241.0\\
%        13.6	&	305.0 & 20.8	&	225.0\\
%        13.7	&	296.0 & 20.9	&	192.0\\
%        13.8	&	286.0 & 21.0	&	188.0\\
%        13.9	&	285.0 & 21.1	&	172.0\\
%        14.0	&	274.0 & 21.2	&	168.0\\
%        14.1	&	264.0 & 21.3	&	169.0\\
%        14.2	&	266.0 & 21.4	&	166.0\\
%        14.3	&	270.0 & 21.5	&	170.0\\
%        14.4	&	255.0 & 21.6	&	174.0\\
%        14.5	&	255.0 & 21.7	&	164.0\\
%        14.6	&	260.0 & 21.8	&	180.0\\
%        14.7	&	251.0 & 21.9	&	179.0\\
%        14.8	&	250.0 & 22.0	&	191.0\\
%        14.9	&	248.0 & 22.1	&	232.0\\
%        15.0	&	253.0 & 22.2	&	300.0\\
%        15.1	&	257.0 & 22.3	&	536.0\\
%        15.2	&	248.0 & 22.4	&	4128.0\\
%        15.3	&	242.0 & 22.5	&	5050.0\\
%        15.4	&	249.0 & 22.6	&	4750.0\\
%        15.5	&	246.0 & 22.7	&	4571.0\\
%        15.6	&	252.0 & 22.8	&	4097.0\\
%        15.7	&	236.0 & 22.9	&	901.0\\
%        15.8	&	234.0 & 23.0	&	244.0\\
%        15.9	&	231.0 & 23.1	&	179.0\\
%        16.0	&	215.0 & 23.2	&	151.0\\
%        16.1	&	217.0 & 23.3	&	145.0\\
%        16.2	&	227.0 & 23.4	&	130.0\\
%        16.3	&	214.0 & 23.5	&	121.0\\
%        16.4	&	217.0 & 23.6	&	126.0\\
%        16.5	&	210.0 & 23.7	&	117.0\\
%        16.6	&	211.0 & 23.8	&	112.0\\
%        16.7	&	206.0 & 23.9	&	110.0\\
%        16.8	&	205.0 & 24.0	&	105.0\\
%        16.9	&	198.0 & 24.1	&	106.0\\
%        17.0	&	203.0 & 24.2	&	107.0\\
%        17.1	&	199.0 & 24.3	&	95.0 \\
%        17.2	&	198.0 & 24.4	&	94.0\\
%        17.3	&	191.0 & 24.5	&	100.0\\
%        17.4	&	192.0 & 24.6	&	91.0\\
%        17.5	&	184.0 & 24.7	&	85.0\\
%        17.6	&	191.0 & 24.8	&	88.0\\
%        17.7	&	188.0 & 24.9	&	83.0\\
%        17.8	&	181.0 & 25.0	&	85.0\\
%        \bottomrule
%    \end{tabular}
%    \caption{Messdaten zur Emission der Kupferröhre.}
%    \label{tab:emiss}
%\end{table}

Zur besseren Visualisierung zeigt Grafik \ref{fig:emission}
das Spektrum. Anhand der Daten lässt sich ablesen, dass die Kupfer 
$K_\alpha$ Linie bei einem Glanzwinkel von $\theta_\alpha = 22.5 \pm 0.1°$ liegt,
während die $K_\beta$ Linie bei $\theta_\beta = 20.2 \pm 0.1 °$ auftritt.\\
Davor ist der Bremsberg zu sehen, der seinen höchsten Punkt bei einem
Winkel von $11.1 \pm 0.1°$ hat. 

\begin{figure}
    \centering
    \includegraphics[width=\textwidth]{emission.pdf}
    \caption{Emissionsspektrum der Cu-Röntgenröhre.}
    \label{fig:emission}
\end{figure}

\noindent Aus den Winkeln lässt sich mithilfe von %\eqref{eqn:}
die jeweilige Wellenlänge des Lichts berechnen. Zudem gilt für Licht
\begin{align*}
    E = h f \\
    f = \frac{c}{\lambda}\\
    \implies E = \frac{h c}{\lambda}
\end{align*}
Daraus folgen für die Linien
\begin{align*}
    E_{K_\alpha} = 8044 \pm 34 \, \mathrm{eV}\\
    E_{K_\beta} = 8910 \pm 40 \, \mathrm{eV}
\end{align*}

\subsection{Transmission}

In Tabelle \ref{tab:trans} sind die aufgenommenen Messdaten zu finden. Bei 
der Messung beträgt die Integrationszeit jeweils $t = 200 \si{\s}$.

\begin{table}
    \centering
    \begin{tabular}{c c c}
        \toprule
        $\alpha$ [°] & $N_0$ [Imp/s] & $N_\text{Al}$ [Imp/s] \\
        \midrule
        7.0	&  226.0 & 113.5 \\ 
        7.1	&  232.0 & 112.0 \\
        7.2	&  240.5 & 112.0 \\
        7.3	&  248.0 & 113.5 \\
        7.4	&  255.0 & 115.0 \\
        7.5	&  262.0 & 113.5 \\
        7.6	&  269.0 & 113.0 \\
        7.7	&  276.0 & 114.5 \\
        7.8	&  281.0 & 114.0 \\
        7.9	&  289.5 & 112.0 \\
        8.0	&  295.0 & 109.5 \\
        8.1	&  300.0 & 109.0 \\
        8.2	&  308.5 & 108.0 \\
        8.3	&  311.0 & 106.0 \\
        8.4	&  317.0 & 104.5 \\
        8.5	&  324.0 & 101.5 \\
        8.6	&  328.5 & 100.0 \\
        8.7	&  332.5 & 100.5  \\
        8.8	&  337.0 & 97.5  \\
        8.9	&  340.5 & 95.0  \\
        9.0	&  348.0 & 92.5  \\
        9.1	&  350.0 & 89.5  \\
        9.2	&  353.0 & 88.0  \\
        9.3	&  356.5 & 84.5 \\
        9.4	&  359.0 & 83.0 \\
        9.5	&  363.5 & 81.0 \\
        9.6	&  367.0 & 78.5 \\
        9.7	&  369.0 & 76.0 \\
        9.8	&  370.5 & 74.0 \\
        9.9	&  375.0 & 72.0 \\
        10.0&  375.5 & 68.5 \\
        \bottomrule
    \end{tabular}
    \caption{Messwerte zur Transmission.}
    \label{tab:trans}
\end{table}

\noindent Unter der Annahme, dass die Zählraten Poisson verteilt sind,
lassen sich die Abweichungen durch $\increment N = \sqrt{N}$ berechnen.
Alle weiteren Fehler werden durch Numeric Python berechnet.\\
Um die Daten zu bereinigen, muss eine Totzeitkorrektur nach %\eqref{}
durchgeführt werden. 
\begin{align*}
    I_0 = \frac{N_0}{1 - \tau N_0}\\
    I_\text{Al} = \frac{N_\text{Al}}{1 - \tau N_\text{Al}}\\
\end{align*}

Wobei $\tau = 90 \si{\micro\s}$ die Totzeit des Geiger-Müller-Zählrohres 
ist.\\
Mithilfe einer linearen Regression wurde die Transmission als Funktion 
der Wellenlänge bestimmt. Abbildung \ref{fig:transmission} zeigt sowohl
die Messdaten als auch die lineare Ausgleichsgerade.
\begin{align*}
    T = a \cdot \lambda + b \\
    a = \left( - 1.518 \pm 0.024 \right) \cdot 10^{10} \\
    b = \left( 1.225 \pm 0.014 \right)\\
\end{align*}

\begin{figure}
    \centering
    \includegraphics[width=\textwidth]{transmission.pdf}
    \caption{Messwerte und Ausgleichsgerade der Transmission.}
    \label{fig:transmission}
\end{figure}

\subsection{Compton-Wellenlänge}

Zur Bestimmung der Compton-Wellenlänge ist die Messung der Transmission
der ungestreuten Röntgenstrahlung und der gestreuten Röntgenstrahlung erforderlich.
Dabei ist $I_0$ die Zählrate ohne Aluminium-Absorber, $I_1$ die mit Absorber
zwischen Röntgenröhre und Plexiglas-Streuer und $I_2$ die Zählrate mit 
Absorber zwischen Plexiglas und Geiger-Müller-Zählrohr. \\
\begin{align*}
    I_0 = 2731 \text{Imp.}\\
    I_1 = 1180 \text{Imp.}\\
    I_2 = 1024 \text{Imp.}\\
\end{align*}
Die Transmission der ungestreuten Röntgenstrahlung ist $T_1 = 0.432$,
die der gestreuten $T_2 = 0.375$.\\
Durch die zuvor berechnete Ausgleichsgerade können diese Transmissionen nun
mit der korrespondierenden Wellenlänge in Verbindung gebracht werden.\\
Diese berechnen sich dabei nach
\begin{equation}
    \lambda = \frac{T - b}{a}
\end{equation}
und lauten $\lambda_1 = \left(5.219 \pm 0.12 \right) \cdot 10^{-11} \si{\m}$ und $\lambda_2 = \left(5.595 \pm 0.13 \right) \cdot 10^{-11} \si{\m}$.\\
Daraus folgt eine Compton-Wellenlänge von $\increment \lambda = \left(3.76 \pm 0.06 \right) \cdot 10^{-11} \si{\m}$.