\section{Theorie}
\label{sec:Theorie}
    Der Compton-Effekt betitelt die Interaktion eines hochenergetischen Photons mit einem 
    niederenergetischen Elektron. Da das Elektron dabei Energie abgibt, verschiebt sich 
    die Frequenz des Photons in den höherfrequenten Raum. Diese Wellenlängendifferenz
    ist unabhängig von der ursprünglichen Wellenlänge und soll in diesem Versuch bestimmt 
    werden.\\
    Dazu wird Röntgenstrahlung erzeugt und an einem Plexiglasquader gestreut. Zu beachten 
    ist allerdings, dass nicht nur Compton (kohärente) Streuung, sondern auch
    elastische (inkohärente) Streuung auf das Photon wirkt. Befindet sich das Elektron 
    vor dem Stoß in Ruhe, kann aus der Energie- und Impulserhaltung auf den Zusammenhang
    \begin{equation*}
        \Delta \lambda = \dfrac{h}{m_e c}(1-\cos{\theta})=\lambda_c(1-\cos{\theta})
    \end{equation*}
    geschlossen werden, wobei $\Delta \lambda$ die Wellenlängendifferenz, $\theta$ der 
    gestreute Winkel (siehe Abb.1), und $\lambda_c=h/(m_e c)$ die Compton-Wellenlänge.
    

