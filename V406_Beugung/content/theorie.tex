\section{Theorie}
\label{sec:Theorie}
Das Ziel dieses Versuchs ist es, einen zusammenhang zwischen Beugungsmuster und dem beugenden Objekt, 
der sogenannten Aperturfunktion, herzustellen.

\paragraph{Parallelspalt}
Es gibt grnerell zwei mögliche Versuchsanordnungen, die Fresnelsche und die Frauenhofersche Lichtbeugung.
Bei der fresnelschen Beugung Liegen sowohl die Lichtquelle als auch der Beobauchtungspunkt im endlichen bereich. 
Das hat zur Folge, dass die Lichtstrahlen die an einem Punkt beobachtet werden verschiedene Beugungswinke haben.
Bei diesem Versuch wird die Frauenhofersche Beugung verwendet, da diese Mathematisch einfacher ist als die
Fresnelsche Beugung. Das bedeutet, dass sowohl die Lichtquelle, als auch Aufpunkt im unendlichen liegen und somit der 
Beugungswinkel für alle Lichtstrahlen eines Beobachtungspunktes gleich ist. 
Die in Z-Richtung einfallende Feldstärke pro Längeneinheit der Wellenfront der Welle wird nun wie folgt beschrieben
\begin{equation}
    \label{eq:1}
    A(z,t) = A_0 exp \left( i(\omega t - \frac{2 \pi z}{\lambda})\right).
\end{equation}
Der verwendete Spalt hat die Breite b und eine Länge die groß gegen b ist. Die begrenzung des Lichtbündels 
ligt also nur in der X-Koordinate. Der Beobachtungsort wird in einer Entfernung zum Spalt gewählt, die sehr groß 
gegen die Spaltbreite ist.
Durch Kombination des Huygensschen Prinzips der Elementarwellen und dem Interferenzprinzips, entsteht Abbildung \ref{fig:a} 
so wie die Annahme, der zu beobachtende Schwingungszustand an einem einzelnen Punkt, ist gegeben durch die Überlagerung 
aller Elementarwellen, die zum selben Zeitpunkt im entsprechenden Punkt ankommen.
\begin{figure}[H]
    \centering
    \includegraphics{Spalt.png}
    \caption{Skizze der Frauenhoferschen Beugung.}
    \label{fig:a}
\end{figure}
Diese einzelnen Elementarwellen sind in Abbildung \ref{fig:a} als Strahlen dargestellt. Weiter ist zu erkennen, dass eben jene 
Elementarwellen die zum selben Zeitpunkt den Beobatungspunkt erreichen und um die Entfernung x im Spalt auseinander liegen, 
sich durch den Wegunterschied s unterscheiden. Da nun von infinitesimale kleinen abständen dx ausgegangen wird, ergibt sich 
für die Amplitund B eine Integration über die gesammte Spaltbreite
\begin{equation}
    \label{eq:2}
    B(z,t,\phi) = A_0 exp\left( i \left( \omega t - \frac{2 \pi z }{\lambda}\right)\right) \int_{0}^{b} exp\left(\frac{2 \pi i x sin \phi}{\lambda}\right) \,dx ,
\end{equation}
woraus sich wiederum nach der Integration und Umformen Gleichung \eqref{eq:3} ergibt.
\begin{equation}
    \label{eq:3}
    B(z,t,\phi) = A_0 exp\left(i \left(\omega t - \frac{2 \pi z}{\lambda}\right)\right) \cdot exp\left(\frac{\pi i b \, sin \phi}{\lambda}\right) \cdot \frac{\lambda}{\pi b \, sin \phi} sin\left(\frac{\pi b \, sin \phi}{\lambda}\right).
\end{equation}
Die beiden Exponentialfunktionen stellen Phasenfunktionen dar, sind daher für die experimentelle Überprüfung irrelevant.
Dafür grgibt sich dann 
\begin{equation}
    \label{eq:4}
    B(\phi) = A_0 b \frac{sin \nu}{\nu},
\end{equation}
mit $\nu := \frac{\pi b \, sin \phi}{\lambda}$.
Die Funktion ist in Abbildung \ref{fig:b} dargestellt und hat die Nullstellen bei 
\begin{equation}
    \label{eq:5}
    sin \phi_n = \pm n\frac{\lambda}{b}.
\end{equation}
\begin{figure}[H]
    \centering
    \includegraphics{Funktion.png}
    \caption{Graphische Darstellung der Amplitudenfunktion $B(\phi)$.}
    \label{fig:b}
\end{figure}
Die Lichhtfrequenz der Lichtwelle $\omega = 10^14$ bis $10^15$ Hz, ist zu hoch für eine unmittelbare Messung, 
weshalb der Mittelwert der Intensität $I(\phi)$ bestimmt wird. Für die Intensität gilt 
\begin{equation}
    \label{eq:6}
    I(\phi) \propto B(\phi)^2.
\end{equation}

\paragraph{Beugung am Doppelspalt}
Die Intensitätsverteilung $I(\phi)$ bei einer Beugung am Doppelspalt kann als Überlagerung zweier Beugungen an je 
einem Spalt der Breite b mit abstand S, wie in Abb \ref{fig:c} dargestellt, betrachtet werden.
Trifft paralleles Licht der Wellenlänge $\lambda$ af den Doppelspalt, ist die Intensitätsverteilung durch 
\begin{equation}
    \label{eq:7}
    I(\phi) \propto B(\phi)^2 = 4 cos^2\left(\frac{\pi s \, sin \phi}{\lambda}\right) \cdot \left(\frac{\lambda}{\pi b \, sin \phi}\right)^2 \cdot sin^2 \left(\frac{\pi b \, sin \phi}{\lambda}\right)
\end{equation}
gegeben.
\begin{figure}[H]
    \centering
    \includegraphics{Doppelspalt.png}
    \caption{Beugung am Doppelspalt.}
    \label{fig:c}
\end{figure}

\paragraph{Frauenhofersche Beugung und Fourier-Transformation}
Die Fouriertransformierte einer Funktion f(x) ist gegeben durch
\begin{equation}
    \label{eq:8}
    g(\xi) :=  \int_{- \infty}^{+ \infty} f(x) \cdot e^{ix \xi} \,dx.
\end{equation}
Es zeigt sich, dass $B(\phi)$ als Forier-Transformierte der Aperturfunktion dargestellt werden kann.
In diesem Beispiel ist die Aperturfunktion $f(x) = A_0$ für 0 $\leq$ x $\leq $b und sonst null.
Daraus ergibt sich durch eben jene Fourier-Transformation
\begin{equation}
    \label{eq:9}
    g(\xi) = \frac{2 A_0}{\xi} exp \left(\frac{i \xi b}{2}\right) sin \frac{\xi b}{2},
\end{equation}
mit
\begin{equation}
    \label{eq:10}
    \xi := \frac{2 \pi \, sin \phi}{\lambda}.
\end{equation}
Da die Fourier-Transformierte Umkehrbar ist, kann ebenso von der Amplitudenfunktion der Gestalt f(x) auf das Brechende 
Objekt geschlossen werden.

