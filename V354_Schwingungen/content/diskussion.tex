\section{Diskussion}
\label{sec:Diskussion}

Zunächst muss bemerkt werden, dass die Kabel nicht vollständig widerstandsfrei sind. 
Die Abweichung der Theoriewerte des Dämpfungswiderstandes von dem effektiven Dämpfungswiderstand
kann dadurch teilweise erklärt werden. Zudem muss der Innenwiderstand des Frequenzoperators 
berücksichtigt werden. Dadurch addiert sich zum Widerstand $R_1$ ein unbekannter Widerstand.
Da $R_{eff}$ größer ist, als $R_1$, kann nur gesagt werden, dass der ermittelte Wert für 
den Dämpfungswiderstandnicht zwingwnd falsch ist.\\

Der Widerstand des aperiodischen Grenzfalls liegt ebenfalls deutlich unter den Theoriewerten.
Auch dies kann durch die oben genannten Faktoren verursacht worden sein.\\

Bei der Messung des Dämpfungswiderstandes wurde außerdem festgestellt, dass sich die Frequenz der Nadelimpuls immer 
wieder ohne Zutun der Versuchsdurchführenden verstellt. Auf den Dämpfungswiderstand 
hat das zunächst keine Einwirkung und danach wurde darauf geachtet, dass sich die 
Frequenz nicht verstellt.\\


