\section{Auswertung}
%-------Pendellänge 1-------
\subsection{Pendellänge 1}
Zunächst wurde das Pendel auf eine Länge $l = 0,824 m$ eingestellt, um die erste Reihe an Messwerten zu nehmen.

\subsubsection{Frei schwingende Pendel}
Da die Zeit für 5 Schwingungen gestoppt wurde, werden die
Werte erst durch fünf geteilt. Dann erhält man die Messwerte
wie in \ref{tab:frei} dargestellt.
\begin{table}
    \centering
    \caption{Messdaten für frei schwingende Pendel.}
    \label{tab:frei}
    \begin{tabular}{c c}
     \toprule
     $T_1$ & $T_2$\\
     \midrule
     1.773 & 1.767 \\
     1.792 & 1.777 \\
     1.768 & 1.8 \\
     1.802 & 1.88\\
     1.779 & 1.791\\
     1.837 & 1.894\\
     1.786 & 1.8 \\
     1.804 & 1.817\\
     1.82 & 1.812 \\
     1.795 & 1.829 \\
     \bottomrule
    \end{tabular}
\end{table}
Als Mittelwert aus den Messwerten ergibt sich eine
Schwingungsdauer $T_1 = 1.796 s$ für das eine und 
$T_2 = 1.817 s$ für das andere Pendel.

\subsubsection{Gleich- und gegenphasige Schwingungen}
Als nächstes sollte das System betrachtet werden, wenn die Feder eingesetzt und die Massen tatsächlich verbunden sind.
Nach Teilung erhalten wir die in Tabelle \ref{tab:g1} aufgelisteten Werte. 
\begin{table}
    \centering
    \caption{Messdaten für gleich- ($T_{+}$) und gegenphasige ($T_{-}$).}
    \label{tab:g1}
    \begin{tabular}{c c}
     \toprule
     $T_{+}$ & $T_{-}$\\
     \midrule
     1.806 & 1.694 \\
     1.814 & 1.66 \\
     1.782 & 1.65 \\
     1.766 & 1.68 \\
     1.812 & 1.656 \\
     1.838 & 1.674 \\
     1.812 & 1.7\\
     1.818 & 1.676\\
     1.794 & 1.638 \\
     1.818 & 1.662 \\
     \bottomrule
    \end{tabular}
\end{table}
Nimmt man jeweils die Mittelwerte, so erhält man $T_{+} = 1.806s$ und $T_{-} = 1.669s$. Aus diesen lässt sich der Koppungsgrad $\kappa$ durch die in der 
Theorie hergeleitete Formel berechnen:
\begin{align}
\kappa = \frac{T_{+}^2 - T_{-}^2}{T_{+}^2 + T_{-}^2}
    = 0.077
\end{align}

\subsubsection{Gekoppelte Schwingung}
Zudem sollte die dritte mögliche Mode genauer betrachtet werden, die gekoppelte Schwingung. Einerseits ist dabei die Schwingung von Bedeutung, andererseits die
Schwebung, also die Zeit, die das System benötigt, bis dasselbe Pendel wieder in Ruhe ist. \\
Gemessene Daten sind in Tabelle \ref{tab:gek1} zu finden.
\begin{table}
    \centering
    \caption{Messdaten für Schwingungs- ($T$) und Schwebungsdauer ($T_{S}$).}
    \label{tab:gek1}
    \begin{tabular}{c c}
     \toprule
     $T$ & $T_{S}$\\
     \midrule
     1.71 & 20.72 \\
     1.664 & 20.85\\
     1.698 & 21.65\\
     1.648 & 21.47\\
     1.692 & 20.72\\
     1.692 & 21.75\\
     1.68 & 20.84\\
     1.67 & 21.0 \\
     1.67 & 20.81\\
     1.67 & 21.18\\
     \bottomrule
    \end{tabular}
\end{table}
Als Mittelwerte erhalten wir $T = 1.678s$ und $T_{S} = 21.099s$. Nun gibt es aber auch eine Formel, mit der sich die Schwebungsdauer theoretisch aus den Schwingungsdauern der
gleich- und gegenphasigen Schwingungen berechnen lassen. Für diesen theoretischen Wert ergibt sich bei unserer Rechnung:
\begin{align}
T_\text{S, theoretisch} = \frac{T_{+} \cdot T_{-}}{T_{+} - T_{-}}
    = 22.141s
\end{align}
Man erkennt, dass sich der theoretisch ermittelte Wert für $T_{S}$ um mehr als eine Sekunde vom gemessenen Wert unterscheidet. 

\subsubsection{Frequenzen}
Zum einen lassen sich die Frequenzen $\omega_{+}$, $\omega_{-}$ aus den Messwerten bestimmen mithilfe der bekannten Formel
\begin{equation}
\omega = \frac{2 \pi}{T}
\end{equation}
berechnen. Daraus erhalten wir für unsere Messdaten $\omega_{+} = 3.478Hz$ und $\omega_{-} = 3.763Hz$. Für die Schwebungsfrequenz $\omega_{S}$ gilt die Formel
\begin{equation}
\omega_{S} = \omega_{+} - \omega_{-}
\end{equation}
Also ergibt sich ein Wert von $-0.284 Hz$. 
Zum anderen wurde in der Theorie je eine Formel für die Eigenfrequenzen 
der gleich- und gegenphasigen Schwingungen hergeleitet. Die daraus berechneten Werte lauten
\begin{align}
\omega_\text{+, theoretisch} = \sqrt{\frac{g}{l}}\\
    = 3.45Hz\\
\omega_\text{-, theoretisch} = \sqrt{\frac{g}{l} + \frac{2 \kappa}{l}}\\
    = 3.476Hz\\
\omega_\text{S, theoretisch} = \omega_\text{+, theoretisch} - \omega_\text{-, theoretisch}\\
    = -0.273Hz
\end{align}
Verglichen mit den theoretischen Werten fällt sofort eine große Ähnlichkeit auf. Die Werte liegen alle sehr nah beieinander, allerdings liegen in der Theorie die
Werte von $\omega_{+}$ und $\omega_{-}$ näher beieinander als die gemessenen Werte.



%-------Pendellänge 2-------
\subsection{Pendellänge 2}
Als zweite Pendellänge haben wir das Pendel auf $0.471 m$ eingestellt.
\subsubsection{Frei schwingende Pendel}
Auch hier wurden die Daten zunächst durch fünf geteilt, dargestellt
sind sie in Tabelle \ref{tab:frei2}. 
\begin{table}
    \centering
    \caption{Messdaten für frei schwingende Pendel mit geringerer Länge.}
    \label{tab:frei2}
    \begin{tabular}{c c}
     \toprule
     $T_1$ & $T_2$\\
     \midrule
     1.406 & 1.424 \\
     1.42 & 1.424 \\
     1.406 & 1.412 \\
     1.406 & 1.432\\
     1.408 & 1.406 \\
     1.368 & 1.432 \\
     1.386 & 1.396 \\
     1.394 & 1.424 \\
     1.394 & 1.432 \\
     1.442 & 1.408 \\
     \bottomrule
    \end{tabular}
\end{table}
Als Mittelwerte ergeben sich $T_1 = 1.403 s$ und $T_2 = 1.419 s$.

\subsubsection{Gleich- und gegenphasige Schwingungen}
Die Feder wurde wieder auf derselben Höhe eingehangen und die ersten beiden Moden wurden betrachtet.
\begin{table}
    \centering
    \caption{Messdaten für gleich- ($T_{+}$) und gegenphasige ($T_{-}$) Mode.}
    \label{tab:g2}
    \begin{tabular}{c c}
     \toprule
     $T_{+}$ & $T_{-}$\\
     \midrule
     1.43 & 1.236 \\
     1.406& 1.272 \\
     1.43 & 1.286 \\
     1.42 & 1.212 \\
     1.402& 1.298 \\
     1.382 & 1.254 \\
     1.456 & 1.262 \\
     1.4240 & 1.27 \\
     1.43 & 1.264 \\
     1.406 & 1.264 \\
     \bottomrule
    \end{tabular}
\end{table}
In Tabelle \ref{tab:g2} sind die Werte aufgeführt, die gemessen wurden. Die Mittelwerte lauten dabei $T_{+} = 1.419s$ und $T_{-} = 1.262s$. Auch hier sollte wieder der Kopplungsgrad 
der Feder bestimmt werden, mit derselben Formel wie oben ergibt er sich zu $\kappa = 0.117$.

\subsubsection{Gekoppelte Schwingung}
Als letzte Mode bleibt wie vorher auch die gekoppelte Schwingung. \ref{tab:gek2} zeigt die Messwerte. \\
$T = 1.264s$ und $T_{S} = 10.513s$ sind die Mittelwerte, wobei wieder $T_{S}$ die Schwebungsdauer ist.
\begin{table}
    \centering
    \caption{Messdaten für Schwingungs- ($T$) und Schwebungsdauer ($T_{S}$).}
    \label{tab:gek2}
    \begin{tabular}{c c}
     \toprule
     $T$ & $T_{S}$\\
     \midrule
     1.244 & 10.58\\
     1.264 & 11.13\\
     1.262 & 10.03\\
     1.288 & 11.35\\
     1.256 & 9.84\\
     1.2 & 11.05\\
     1.244 & 10.43\\
     1.344 & 10.1\\
     1.25 & 10.29\\
     1.288 & 10.33\\
     \bottomrule
    \end{tabular}
\end{table}
Durch Berechnung mit der Formel 
\begin{equation}
T_\text{S, theoretisch} = \frac{T_{+} \cdot T_{-}}{T_{+} - T_{-}}
\end{equation}
erhalten wir den Theoriewert $T_\text{S, theoretisch} = 11.416s$, der auch wieder etwa eine Sekunde höher ist als das gemessene Ergebnis.

\subsubsection{Frequenzen}
Zuletzt sollen noch die Frequenzen für diese zweite Pendellänge betrachtet werden.
$\omega_{+}$ und $\omega_{-}$ werden wieder mithilfe der Messwerte aus der Formel
\begin{equation}
\omega = \frac{2 \pi}{T}
\end{equation}
berechnet. Es ergeben sich $\omega_{+} = 4.428Hz$ und $\omega_{-} = 4.98Hz$. Für die Schwebungsfrequenz $\omega_{S}$ ergibt sich mit 
\begin{equation}
\omega_{S} = \omega_{+} - \omega_{-}
\end{equation}
ein Wert von $-0.54 Hz$.
Die Theoriewerte lauten
\begin{align}
\omega_\text{+, theoretisch} = \sqrt{\frac{g}{l}}\\
    = 4.563Hz\\
\omega_\text{-, theoretisch} = \sqrt{\frac{g}{l} + \frac{2 \kappa}{l}}\\
    = 4.617Hz\\
\omega_\text{S, theoretisch} = \omega_{+} - \omega_{-}\\
    = -0.054Hz
\end{align}
Auffällig ist insbesondere die stark unterschiedliche Schwebungsfrequenz.