\section{Diskussion}
Generell lässt sich sagen, dass wie erwartet die Schwingungsdauern der gegenphasigen Schwingung kürzer waren. Allerdings haben wir auch erwartet, hohe 
Messunsicherheiten festzustellen und somit Diskrepanzen zu den theoretisch ermittelten Werten. Dies liegt insbesondere am menschlichen Fehler, da es nicht
leicht ist, die maximale Auslenkung genau zu treffen; hinzu kommt, dass man bei der Anregung in gleich- oder gegenphasige Mode kaum denselben Winkel treffen
konnte und somit die Mode nicht perfekt anregen konnte. Außerdem ist unausweichlich auch immer eine geringe Bewegung zur Wand hin und von der Wand weg hinzu 
gekommen, wodurch ein Teil der Energie der Pendelbewegung verloren gegangen ist und kein perfekter harmonischer Oszillator vorlag, so wie man ihn in der Theorie
angenommen hat. Reibung ist natürlich in der Praxis nicht zu vernachlässigen, die in die theoretische Rechnung auch nicht eingeflossen ist. So kämen noch Einflüsse
von Luftreibung hinzu, sowie die Reibung der Drehachse.\\
Genau diese Diskrepanz zwischen unseren Messwerten und der Theorie haben wir auch gesehen, konkret gehen wir nun auf Schwebungsdauer und die Frequenzen ein.

\subsection{Schwebungsdauer}
Beim Vergleich der theoretisch berechneten Schwebungsdauern mit dem Mittelwert der Messungen fällt ein verhältnismässig großer Unterschied von $ s$ auf. 
Dafür gibt es verschiedene Gründe. Zum einen ist die Messung der Schwebungsdauer nicht komplett genau möglich, da es kaum möglich ist, die Zeit genau zu stoppen,
und da 10 Messungen diese statistische Messunsicherheit nicht gänzlich aufheben können, ist so der Fehler für den gemessenen Wert recht hoch.
Zum anderen ist auch die Bestimmung über die theoretische Formel nicht fehlerfrei, da diese die Messwerte für $T_{+}$ und $T_{-}$ verwendet und auch diese mit einer
Messunsicherheit behaftet sind. Es ist besonders schwer, bei der Auslenkung in gleich- oder gegenphasiger Mode denselben Winkel zu treffen. Dadurch ist diese Mode
nicht einwandfrei angeregt und die Messung daher ungenau.\\
Hinzu kommt, dass die nicht genau auf einer Länge eingestellt sind, nur gemäß der Aufgabenstellung so, dass die Unterschiede in den Schwingungsdauern geringer sind
als die berechnete Messungenauigkeit. Auch schon durch die Bestimmung der Länge mit einem einfachen Maßband liegt eine gewisse systematische Messunsicherheit vor.
Gemeinsam ergeben diese Unsicherheiten dann den Unterschied in der Schwebungsdauer. Bei der zweiten Pendellänge ergibt sich aus denselben Gründen eine Differenz
zwischen Mess- und Theoriewert von $ s$.

\subsection{Frequenzen}
Auch die Theorie- und Messwerte der Frequenzen sind nicht gleich; bei der ersten Pendellänge sieht man Unterschiede von $ s$, also alle sehr nah beieinander. Bei
der zweiten Pendellänge fällt besonders der Unterschied zwischen $\omega_\text{s, theoretisch}$ und $\omega_{s}$ auf: die Differenz zwischen den Werten liegt bei
$ s$. Wie zuvor gesagt fallen sowohl statistische als auch systematische Messunsicherheiten als Grund ins Gewicht, hier ist insbesondere ... zu erwähnen.