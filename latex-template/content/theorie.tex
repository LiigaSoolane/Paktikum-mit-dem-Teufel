\section{Theorie}
\label{sec:Theorie}
\subsection{Das Fadenpendel}
    Ein Fadenpendel besteht aus einem Pendelkörper mit Masse m, der idealerweise 
    an einem masselosen Faden; in unserem Fall an einer Metallstange befestigt ist.
    Wird der Pendelkörper aus der Ruhelage ausgelenkt, so wirkt aufgrund der 
    Gravitation die rücktreibende Kraft $\vec{F}=m\cdot \vec{g}$. Dadurch entsteht  
    ein Drehmoment $M= D_p \cdot \sin{\alpha}$ mit $D_p= m\cdot g\cdot l$. Für kleine Auslenkungen 
    kann die Kleinwinkelnäherung $\sin{\alpha}=\alpha$ verwendet werden. %\nocite{wiki}
    Unter Berücksichtigung des Trägheitsmomentes $J=m\cdot l²$ ergibt sich folgende Bewegungsgleichung 
    für den Pendelkörper
    \begin{equation}
        D_p \cdot \sin{\alpha}+J\cdot \ddot \alpha=0
    \end{equation}
    Für kleine Winkel kann das Fadenpendel also als harmonischer Oszillator angenähert werden.
    Die Lösung dieser Gleichung lautet
    \begin{equation}
        \alpha=a\cdot \sin{\omega t} + b\cdot \cos{\omega t}
    \end{equation}
    mit Schwingungsfrequenz
    \begin{equation}
        \omega= \sqrt{\dfrac{D_p}{J}}=\sqrt{\dfrac{g}{l}}
    \end{equation}
    Bei kleinen Auslenkungen haben daher weder die Masse des Pendelkörpers noch der
    Auslenkwinkel Einfluss auf die Periodendauer $T=\dfrac{2\pi}{\omega}$.

\subsection{Verbunden}
    Durch die Kopplung von zwei Fadenpendeln durch eine Feder kann Energie des einen 
    Fadenpendels auf das andere übertragen werden. Die zusätzlichen Drehmomente der Pendel
    lauten: $M_1=D_F(\alpha_2-\alpha_1),\ M_2=D_F(\alpha_1-\alpha_2)$
    \subsection{Gleichsinnige Schwingungen}
        Im Fall von gleichsinnigen Schwingungen findet keine Energieübertragung zwischen
        den beiden Systemen statt, da die Feder im Idealfall weder ge- noch entspannt wird.
        Bei zwei Pendeln der selben Länge ist zu erwarten, dass sich die Periodendauer durch
        die Kopplung nicht ändern; vorrausgesetzt, die Aulenkungen zu Beginn betragen exakt 
        den selben Wert.
    \subsection{Gegensinnige Schwingungen}
        Im Fall von gegensinnigen Schwingungen verstärkt die Feder die rücktreibenden Kräfte
        der einzelnen Pendel. Dadurch wird die Periodendauer verkürzt. 
    \subsection{Gekoppelte Schwingungen}
        Wird zu Beginn das eine Pendel ausgelenkt, das andere jedoch in der Ruheposition 
        belassen, so wird Energie langsam hin und her übertragen (Dies gilt zumindestens, wenn die
        Rückstellkraft der Feder gering ist im Vergleich zur Rückstellkraft der Pendel).
