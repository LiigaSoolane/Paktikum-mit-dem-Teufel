\section{Auswertung}
\label{sec:Auswertung}

\subsection{Bestimmung der Metalle}
  Zur Bestimmung der Metalle, aus denen die untersuchten Stäbe bestehen, werden
  zunächst die Eigenschäften der Stäbe aufgenommen. Sie sind in Tabelle \ref{tab:mass}
  zu finden.

  \begin{table}
    \centering
    \caption{Vermessung des runden Stabs.}
    \label{tab:mass}
    \begin{tabular}{c c c c}
      \toprule
      \multicolumn{2}{c}{runder Stab} & \multicolumn{2}{c}{quadratischer Stab}\\
      \cmidrule(lr){1-2}\cmidrule(lr){3-4}{Größe} & {Messwert } & {Größe} & {Messwert}\\
      \midrule
      Länge & $0.58 \si{\m}$& Länge & $0.60 \si{\m}$\\
      Durchmesser & $0.01 \si{\m}$& Breite und Höhe & $1.025 \cdot 10^{-2} \si{\m}$\\
      Masse & $356 \si{\g}$ & Masse & $535.7 \si{\g}$\\
      \bottomrule
    \end{tabular}
  \end{table}

  \noindent Aus den Massen und Volumen können nach $\rho = M/V$ die Dichten $\rho$ der 
  Metalle berechnet werden. Es ergeben sich
  \begin{align*}
    \rho_{\symup{rund}} = 7.808 \si{\g\per\centi\m\cubed}\\
    \rho_{\symup{quad}} = 8.498 \si{\g\per\centi\m\cubed}
  \end{align*}
  woraus sich der runde Stab als Eisen identifizieren lässt (Dichte nach \cite{dichten}: $\rho_{\symup{Eisen}} = 7.87 \si{\g\per\centi\m\cubed}$)
  und der Stab quadratischen Querschnitts als Messing (Dichte nach \cite{dichten}: $\rho_{\symup{Messing}} = 8.5 \si{\g\per\centi\m\cubed}$).\\

\subsection{Berechnung der Flächenträgheitsmomente}
  Bevor mit der Bestimmung der Elastizitätsmodule begonnen werden kann, müssen die Flächenträgheitsmomente der beiden 
  Stäbe berechnet werden.\\
  Für den runden Eisenstab lässt sich das Integral in Gleichung \eqref{eqn:} gut in Polarkoordinaten 
  umformen. Dann gilt
  \begin{align*}
    I_{\symup{rund}} &= \int_0^{2 \pi} \int_0^{\frac{d}{2}} \left( r \cos(\phi)\right)^2 r \symup{d}r \symup{d}\phi \\
    &= \frac{1}{4} r^4|_{0}^{ \frac{d}{2} } \int_0^{2 \pi} \cos^2(\phi) \symup{d}\phi \\
    &= \frac{d^4}{64} \int_{0}^{2 \pi} \frac{1}{2} + \frac{1}{2} \cos{2 \phi} \symup{d}\phi\\
    &= \frac{d^4}{64} \pi 
  \end{align*}
  mit dem Durchmesser $d$ des Eisenstabes.\\

  \noindent Dagegen lässt sich das Flächenträgheitsmoment des Messingstabs in kartesischen Koordinaten berechnen. Es gilt
  \begin{align*}
    I_{\symup{quad}} &= \int_{- \frac{a}{2}}^{ \frac{a}{2} } \int_{ - \frac{a}{2}}^{ \frac{a}{2}} y^2 \symup{d}y \symup{d}x \\
    &= x|_{ - \frac{a}{2}}^{ \frac{a}{2}} \int_{ - \frac{a}{2}}^{ \frac{a}{2}} y^2 \symup{d}y\\
    &= a \cdot \frac{1}{3} y^3|_{ - \frac{a}{2}}^{ \frac{a}{2}}\\
    &= \frac{a^4}{12}
  \end{align*}
  mit der Breite und Höhe des $a$ des Messingstabes.

\subsection{Einseitige Einspannung}
  Nun können die Elastizitätsmodule bestimmt werden. Tabellen \ref{tab:rund} und \ref{tab:eck} 
  zeigen die aufgenommenen Messwerte.\\

  \begin{table}
    \centering
    \caption{Messung der Biegung des Messingstabs bei einseitiger Einspannung.}
    \label{tab:eck}
    \begin{tabular}{c c}
      \toprule
      Abstand von der Einspannung $\mathbin{/} \si{\centi\m}$ & Durchbiegung $\mathbin{/} \si{\milli\m}$\\
      \midrule
      3  & 0\\
      8  & 0.001\\
      13 & 0.03\\
      16 & 0.005\\
      21 & 0.11\\
      26 & 0.42\\
      29 & 0.71\\
      31 & 0.94\\
      33 & 1.21\\
      35 & 1.415\\
      37 & 1.64\\
      39 & 1.865\\
      41 & 2.125\\
      43 & 2.355\\
      45 & 2.58\\
      47 & 2.83\\
      49 & 3.02\\
      51 & 3.25\\
      53 & 3.31\\
      \bottomrule
    \end{tabular}
  \end{table}

  \begin{table}
    \centering
    \caption{Messung der Biegung des Eisenstabs bei einseitiger Einspannung.}
    \label{tab:rund}
    \begin{tabular}{c c}
      \toprule
      Abstand von der Einspannung $\mathbin{/} \si{\centi\m}$ & Durchbiegung $\mathbin{/} \si{\milli\m}$\\
      \midrule
      3  & 0.01\\
      8  & 0.1\\
      13 & 0.29\\
      18 & 0.51\\
      23 & 0.74\\
      25 & 0.88\\
      27 & 0.965\\
      29 & 1.14\\
      31 & 1.25\\
      33 & 1.41\\
      35 & 1.52\\
      37 & 1.64\\
      39 & 1.825\\
      41 & 2.02\\
      43 & 2.1\\
      45 & 2.275\\
      47 & 2.385\\
      49 & 2.62\\
      51 & 2.765\\
      \bottomrule
    \end{tabular}
  \end{table}
  \FloatBarrier

  \noindent Um zu einer linearen Darstellung von D zu gelangen, werden diese Daten in den Term
  \begin{equation*}
    L x^2 - \frac{x^3}{3}
  \end{equation*}
  aus Gleichung \eqref{eqn:} eingesetzt. Dann lässt sich eine lineare Regression durchführen, die 
  für die Darstellung $D(x) = a \cdot x + b$ die Parameter
  \begin{align*}
    a_{\symup{rund}} = \left(2.997 \pm 0.024 \right) \cdot 10^{-2}\\
    b_{\symup{rund}} = \left(3.900 \pm 1.300 \right) \cdot 10^{-5}\\
  \end{align*}
  für den runden Stab und die Parameter
  \begin{align*}
    a_{\symup{quad}} = \left( 3.660 \pm 0.130 \right) \cdot 10^{-2}\\
    b_{\symup{quad}} = \left(0.040 \pm 0.008 \right) \cdot 10^{-2}\\
  \end{align*}
  für den Stab mit quadratischem Querschnitt liefert. Die Fehler wurden dabei nach 
  \begin{equation}
    \Delta
    \label{eqn:linerr}
  \end{equation}
  berechnet. Aus den Steigungen der Geraden lässt sich das Elastizitätsmodul bestimmen, wenn man den 
  entsprechenden Teil aus \ref{} umstellt. Es folgt
  \begin{align*}
    a &= \frac{F}{2 E I}\\
    \implies E &= \frac{F}{2 a I}
  \end{align*}
  mit der Kraft $F$, die auf den jeweiligen Stab wirkt und dem zuvor berechneten Flächenträgheitsmoment $I$.
  Die Kraft entspricht der Gewichtskraft $F = m \cdot g$ mit Gravitationsbeschleunigung $g = $ \cite{}
  und dem jeweils verwendeten Gewicht $m$. \\
  Die Messwerte und die linearen Ausgleichsgeraden sind in Abbildung \ref{fig:plots} dargestellt.
  Es ergeben sich nach Einsetzen
  \begin{align*}
    E_{\symup{Eisen, einseitig}} = \left( 1.761 \pm 0.014\right)\cdot 10^{11} \si{\newton\per\m}\\
    E_{\symup{Messing}} = \left( 0.945 \pm 0.035 \right)\cdot 10^{11} \si{\newton\per\m}.\\ %%%%%%%%% argenti vive
  \end{align*}

  \begin{figure}
    \centering
    \includegraphics[width=\textwidth]{plots_einseitig.pdf}
    \caption{Messung der Durchbiegung beider Stäbe bei einseitiger Auflage.}
    \label{fig:plots}
  \end{figure}
  

\subsection{Beidseitige Einspannung}
  Bei der Untersuchung des Verhaltens der Biegung unter beidseitiger Einspannung wurde nur der 
  Eisenstab betrachtet. \\
  Die aufgenommenen Messwerte sind in Tabellen zu finden. %%%%%%%%% add here

  \begin{table}
    \centering
    \caption{Messwerte bei beidseitiger Auflage, Bereich $0 \leq x \leq \frac{L}{2}$.}
    \label{tab:rechts}
    \begin{tabular}{c c}
      \toprule
      Abstand $x \mathbin{/} \si{\centi\m}$ & Durchbiegung $D \mathbin{/} \si{\milli\m}$ \\
      \midrule
      3  & 0\\
      8  & 0\\
      13 & 0.035\\
      17 & 0.075\\
      19 & 0.14\\
      21 & 0.16\\
      23 & 0.2\\
      25 & 0.19\\
      27 & 0.21\\
      \bottomrule
    \end{tabular}
  \end{table}

  \begin{table}
    \centering
    \caption{Messwerte bei beidseitiger Auflage, Bereich $\frac{L}{2} \leq x \leq L$.}
    \label{tab:links}
    \begin{tabular}{c c}
      \toprule
      Abstand $x \mathbin{/} \si{\centi\m}$ & Durchbiegung $D \mathbin{/} \si{\milli\m}$ \\
      \midrule
      29 & 0.18\\
      31 & 0.2\\
      33 & 0.185\\
      35 & 0.16\\
      37 & 0.15\\
      39 & 0.13\\
      44 & 0.12\\
      49 & 0.01\\
      54 & 0\\
      \bottomrule
    \end{tabular}
  \end{table}
  \FloatBarrier

  \noindent Es werden also der Bereich rechts von der Auflage des Gewichts und links davon 
  getrennt betrachet.\\
  Werden die nichtlinearen Teile in \ref{} und \ref{} berechnet, lassen sich wieder lineare 
  Regressionen durchführen, aus denen das Elastizitätsmodul von Eisen nach 
  \begin{align*}
    a &= \frac{F}{48 E I} \\
    E &= \frac{F}{48 a I}
  \end{align*}
  bestimmt werden kann.\\
  Die Parameter ergebens sich für den Bereich $\leq \frac{L}{2}$ zu 
  \begin{align*}
    a_{\symup{rechts}} &= \left( 0.154 \pm 0.026 \right) \cdot 10^{-2}\\
    b_{\symup{rechts}} &= \left( 0.009 \pm 0.004 \right) \cdot 10^{-2}\\
  \end{align*}
  und für den Bereich $\geq \frac{L}{2}$ zu 
  \begin{align*}
    a_{\symup{links}} = \left( 0.130 \pm 0.012 \right) \cdot 10^{-2}\\
    b_{\symup{links}} = \left( 0.041 \pm 0.017 \right) \cdot 10^{-2}.\\
  \end{align*}
  Messwerte sowie Fit sind in Abbildungen dargestellt.
  Für das Elastizitätsmodul ergibt sich
  \begin{align*}
    E_{\symup{Eisen, rechts}} = \left( 2.030 \pm 0.340 \right) \cdot 10^{11} \si{\newton\per\m}\\
    E_{\symup{Eisen, links}} = \left( 2.410 \pm 0.230 \right) \cdot 10^{11} \si{\newton\per\m}.\\
  \end{align*}

  \begin{figure}
    \centering
    \includegraphics[width=\textwidth]{plot_rechts.pdf}
    \caption{Messung der Durchbiegung bei beidseitiger Auflage, Bereich $\leq \frac{L}{2}$.}
    \label{fig:rechts}
  \end{figure}

  \begin{figure}
    \centering
    \includegraphics[width=\textwidth]{plot_links.pdf}
    \caption{Messung der Durchbiegung bei beidseitiger Auflage, Bereich $\geq \frac{L}{2}$.}
    \label{fig:links}
  \end{figure}