\section{Auswertung}
\label{sec:Auswertung}

\subsection{Vanadium}
Als Erstes sollte die Untergrundrate ermittelt werden, dazu wurde in 
$300 \si{\s}$-Intervallen die Zerfallsrate gemessen. So ergaben sich
$N_U = {129, 143, 144, 136, 126, 158}$.

Anschließend wurde der Zerfall von $^{52}V$ gemessen. Dabei handelt sich um 
ein Isotop mit einfachem Zerfall. Als Zeitintervall wurde $\increment t = 30 \si{\s}$
geewählt, sodass sich die in Tabelle \ref{tab:vanwerte} ergaben. Die Fehler wurden dabei 
durch die Poisson-Verteilung $\increment N = \sqrt{N}$ ermittelt.

\begin{table}
 \centering
 \caption{Messdaten für den Zerfall von Vanadium.}
 \label{tab:vanwerte}
 \begin{tabular}{c c c}
  \toprule
  {$t \mathbin{/} \si{\s}$} & {$N_{\increment t, \text{gem}}$} & {$\increment N_{\increment t, \text{gem}}$}\\
  \midrule
  30.0                 & 189.0                & 13.74  \\
  60.0                 & 197.0                & 14.03  \\
  90.0                 & 150.0                & 12.24  \\
  120.0                & 159.0                & 12.60  \\
  150.0                & 155.0                & 12.44  \\
  180.0                & 132.0                & 11.48  \\
  210.0                & 117.0                & 10.81  \\
  240.0                & 107.0                & 10.34  \\
  270.0                & 94.0                 & 9.695  \\
  300.0                & 100.0                & 10.0   \\
  330.0                & 79.0                 & 8.888  \\
  360.0                & 69.0                 & 8.306  \\
  390.0                & 81.0                 & 9.0    \\
  420.0                & 46.0                 & 6.782  \\
  450.0                & 49.0                 & 7.0    \\
  480.0                & 61.0                 & 7.810  \\
  510.0                & 56.0                 & 7.483  \\
  540.0                & 40.0                 & 6.324  \\
  570.0                & 45.0                 & 6.708  \\
  600.0                & 32.0                 & 5.656  \\
  630.0                & 27.0                 & 5.196  \\
  660.0                & 43.0                 & 6.557  \\
  690.0                & 35.0                 & 5.916  \\
  720.0                & 19.0                 & 4.358  \\
  750.0                & 28.0                 & 5.291  \\
  780.0                & 27.0                 & 5.196  \\
  810.0                & 36.0                 & 6.0    \\
  840.0                & 25.0                 & 5.0    \\
  870.0                & 29.0                 & 5.385  \\
  900.0                & 18.0                 & 4.242  \\
  930.0                & 17.0                 & 4.123  \\
  960.0                & 24.0                 & 4.898  \\
  990.0                & 21.0                 & 4.582  \\
  1020.0               & 25.0                 & 5.0    \\
  1050.0               & 21.0                 & 4.582  \\
  1080.0               & 24.0                 & 4.898  \\
  1110.0               & 25.0                 & 5.0    \\
  1140.0               & 17.0                 & 4.123  \\
  1170.0               & 20.0                 & 4.472  \\
  1200.0               & 19.0                 & 4.358  \\
  1230.0               & 20.0                 & 4.472  \\
  1260.0               & 18.0                 & 4.242  \\
  1290.0               & 16.0                 & 4.0    \\
  1320.0               & 17.0                 & 4.123  \\
  \bottomrule
 \end{tabular}
\end{table}

\noindent Mithilfe des Mittelwertes der Messung für den Untergrund $N_U = \pm $
folgen die wahren Werte für $N_{\increment t}$ wie in Tabelle \ref{tab:vanwahr} 
aufgeführt.

\begin{table}
 \centering
 \caption{Messwerte von Vanadium ohne Untergrundrate.}
 \label{tab:vanwahr}
 \begin{tabular}{c c c}
  \toprule
  {$t \mathbin{/} \si{\s}$} & {$N_{\increment t}$} & {$\increment N_{\increment t}$}\\
  \midrule
  30.0    & 175.066  & 13.831  \\
  60.0    & 183.066 & 14.118  \\
  90.0    & 136.066  & 12.341  \\
  120.0   & 145.066  & 12.701  \\
  150.0   & 141.066  & 12.542  \\
  180.0   & 118.066  & 11.589  \\
  210.0   & 103.066  & 10.923  \\
  240.0   & 93.066  & 10.455  \\
  270.0   & 80.066  & 9.814  \\
  300.0   & 86.066  & 10.115  \\
  330.0   & 65.066  & 9.017\\
  360.0   & 55.066  & 8.445  \\
  390.0   & 67.066  & 9.128  \\
  420.0   & 32.066  & 6.951  \\
  450.0   & 35.066  & 7.163  \\
  480.0   & 47.066  & 7.957  \\
  510.0   & 42.066  & 7.636  \\
  540.0   & 26.066  & 6.505  \\
  570.0   & 31.066  & 6.879  \\
  600.0   & 18.066  & 5.858  \\
  630.0   & 13.066  & 5.414  \\
  660.0   & 29.066  & 6.732  \\
  690.0   & 21.066  & 6.109  \\
  720.0   & 5.066  & 4.617  \\
  750.0   & 14.066  & 5.506  \\
  780.0   & 13.066  & 5.414  \\
  810.0   & 22.066  & 6.190  \\
  840.0   & 11.066  & 5.227  \\
  870.0   & 15.066  & 5.596  \\
  900.0   & 4.066  & 4.508  \\
  930.0   & 3.066  & 4.395  \\
  960.0   & 10.06  & 5.130  \\
  990.0   & 7.066  & 4.829  \\
  1020.0  & 11.06  & 5.227  \\
  1050.0  & 7.066  & 4.829  \\
  1080.0  & 10.06  & 5.130  \\
  1110.0  & 11.06  & 5.227  \\
  1140.0  & 3.066  & 4.395  \\
  1170.0  & 6.066  & 4.724  \\
  1200.0  & 5.066  & 4.617  \\
  1230.0  & 6.066  & 4.724  \\
  1260.0  & 4.066  & 4.508  \\
  1290.0  & 2.066  & 4.280  \\
  1320.0  & 3.066  & 4.395  \\
  \bottomrule
 \end{tabular}
\end{table}

\noindent Aus diesen Werten wurde mithilfe einer linearen Regression
eine Kurve für den Zerfall von Vanadium errechnet, nach Formel \eqref{eqn:ln}
aus der Theorie. Es gilt:
\begin{align*}
    \lambda = (0.323 \pm 0.011) \cdot 10^(-2) \\
    c = 5.32 \pm 0.04 \\
    ln(N_{\increment t}) = c - \lambda t 
\end{align*}
Aus dem Zerfallsparameter $\lambda$ lässt sich die Halbwertszeit $T_V$ von
Vanadium berechnen nach \eqref{eqn:t}.
\begin{equation*}
    T_V = 215+/-7 \si{\s}
\end{equation*}
Eine genauere Methode nimmt nicht alle Werte in die lineare Regression auf,
dazu wurden alle Wertepaare nach der doppelten Halbwertszeit vernachlässigt.
Es folgt
\begin{align*}
    \lambda_2 = (0.333 \pm 0.022) \cdot 10^(-2) \\
    c_2 = 5.34 \pm 0.05 \\
    ln(N_{\increment t}) = c_2 - \lambda_2 t \\
    T_V,2 = 208+/-13
\end{align*}
Abbildung \ref{fig:vankurve}
zeigt die Messwerte ohne Untergrund sowie die beiden linearen Regressionen.

\begin{figure}
 \centering
 \includegraphics[width=\textwidth]{Vanadium.pdf}
 \caption{Messdaten und Fits für den Zerfall von Vanadium.}
 \label{fig:vankurve}
\end{figure}

\subsection{Rhodium}

Für die Messung des Zerfalls von Rhodium wurde ein Zeitintervall von 
$\increment t = 15 \si{\s}$ gewählt. Die Messdaten sind in Tabelle \ref{tab:rhowerte}
aufgelistet.

\begin{table}
 \centering
 \caption{Messwerte von Rhodium.}
 \label{tab:rhowerte}
 \begin{tabular}{c c c}
  \toprule
  {$t \mathbin{/} \si{\s}$} & {$N_{\increment t, \text{gem}}$} & {$\increment N_{\increment t, \text{gem}}$}\\
  \midrule
  15.0    & 667.0  & 25.826 \\
  30.0    & 585.0  & 24.186  \\  
  45.0    & 474.0  & 21.771  \\  
  60.0    & 399.0  & 19.974  \\  
  75.0    & 304.0  & 17.435  \\  
  90.0    & 253.0  & 15.905  \\  
  105.0   & 213.0  & 14.594  \\  
  120.0   & 173.0  & 13.152  \\  
  135.0   & 152.0  & 12.328  \\  
  150.0   & 126.0  & 11.224  \\  
  165.0   & 111.0  & 10.535  \\  
  180.0   & 92.0   & 9.591  \\  
  195.0   & 79.0   & 8.888  \\  
  210.0   & 74.0   & 8.602  \\  
  225.0   & 60.0   & 7.745  \\  
  240.0   & 52.0   & 7.211  \\  
  255.0   & 56.0   & 7.483  \\  
  270.0   & 53.0   & 7.280  \\  
  285.0   & 41.0   & 6.403  \\  
  300.0   & 36.0   & 6.0     \\  
  315.0   & 37.0   & 6.082  \\  
  330.0   & 32.0   & 5.656  \\  
  345.0   & 36.0   & 6.0    \\  
  360.0   & 38.0   & 6.164  \\  
  375.0   & 34.0   & 5.830  \\  
  390.0   & 40.0   & 6.324  \\  
  405.0   & 21.0   & 4.582  \\  
  420.0   & 35.0   & 5.916  \\  
  435.0   & 33.0   & 5.744  \\  
  450.0   & 36.0   & 6.0     \\  
  465.0   & 20.0   & 4.472  \\  
  480.0   & 24.0   & 4.898  \\  
  495.0   & 30.0   & 5.477  \\  
  510.0   & 30.0   & 5.477  \\  
  525.0   & 26.0   & 5.099  \\  
  540.0   & 28.0   & 5.291  \\  
  555.0   & 23.0   & 4.795  \\  
  570.0   & 20.0   & 4.472  \\  
  585.0   & 28.0   & 5.291  \\  
  600.0   & 17.0   & 4.123  \\  
  615.0   & 26.0   & 5.099  \\  
  630.0   & 19.0   & 4.358  \\  
  645.0   & 13.0   & 3.605  \\  
  660.0   & 17.0   & 4.123 \\
  \bottomrule
 \end{tabular}
\end{table} 

Nach abziehen der auf ein Zeitintervall von 15 Sekunden normierten Untergrundrate
bleiben die in Tabelle \ref{tab:rhowahr} 
gezeigten Daten.

\begin{table}
 \centering
 \caption{Messwerte von Rhodium ohne Untergrund.}
 \label{tab:rhowahr}
 \begin{tabular}{c c c}
  \toprule
  {$t \mathbin{/} \si{\s}$} & {$N_{\increment t}$} & {$\increment N_{\increment t}$}\\
  \midrule
  15.0   & 660.033  & 25.848    \\
  30.0   & 578.033  & 24.210  \\
  45.0   & 467.033  & 21.798  \\
  60.0   & 392.033  & 20.004  \\
  75.0   & 297.033  & 17.468  \\
  90.0   & 246.033  & 15.942  \\
  105.0  & 206.033  & 14.634  \\
  120.0  & 166.033  & 13.197  \\
  135.0  & 145.033  & 12.375  \\
  150.0  & 119.033  & 11.276  \\
  165.0  & 104.033  & 10.590  \\
  180.0  & 85.033  & 9.652  \\
  195.0  & 72.033  & 8.953  \\
  210.0  & 67.033  & 8.669  \\
  225.0  & 53.033  & 7.820  \\
  240.0  & 45.033  & 7.291  \\
  255.0  & 49.033  & 7.560  \\
  270.0  & 46.033  & 7.359  \\
  285.0  & 34.033  & 6.493  \\
  300.0  & 29.033  & 6.095  \\
  315.0  & 30.033  & 6.177  \\
  330.0  & 25.033  & 5.758  \\
  345.0  & 29.033  & 6.095  \\
  360.0  & 31.033  & 6.257  \\
  375.0  & 27.033  & 5.929  \\
  390.0  & 33.033  & 6.415  \\
  405.0  & 14.033  & 4.707  \\
  420.0  & 28.033  & 6.013  \\
  435.0  & 26.033  & 5.844  \\
  450.0  & 29.033  & 6.095  \\
  465.0  & 13.033  & 4.600  \\
  480.0  & 17.033  & 5.016  \\
  495.0  & 23.033  & 5.582  \\
  510.0  & 23.033  & 5.582  \\
  525.0  & 19.033  & 5.211  \\
  540.0  & 21.033  & 5.400  \\
  555.0  & 16.033  & 4.915  \\
  570.0  & 13.033  & 4.600  \\
  585.0  & 21.033  & 5.400  \\
  600.0  & 10.033  & 4.261  \\
  615.0  & 19.033  & 5.211  \\
  630.0  & 12.033  & 4.490  \\
  645.0  & 6.033   & 3.763  \\
  660.0  & 10.033  & 4.261 \\
  \bottomrule
 \end{tabular}
\end{table}

\noindent Nach Auftragen der Messwerte halblogarithmisch gegen die Zeit wie in Abbildung \ref{fig:messf},
ist eine Struktur erkennbar. Ab der Zeit $t_l = 270 \si{\s}$ ist nur noch der Zefall vom langlebigeren $^{104i}Rh$
von Bedeutung, während der Zerfall des kurzlebigeren Isotopes $^{104}Rh$ bis zur Zeit $t_s = 150 \si{\s}$ vorherrschend ist.

\begin{figure}
 \centering
 \includegraphics[width=\textwidth]{werte.pdf}
 \caption{Messwerte des Rhodiumzerfalls mit Fehlern.}
 \label{fig:messf}
\end{figure}

\noindent Für $t > t_l$ lässt sich die Kurve des langlebigen Zerfalls extrapolieren, sodass 
sie dargestellt werden kann mit
\begin{align*}
    \lambda_l = (0.323 \pm 0.11) \cdot 10^{-2}\\
    c_l = 4.35 \pm 0.17 \\
    ln(N_{langlebig}) = c_l - \lambda_l \cdot t
\end{align*}

Nach Abziehen des so erhaltenen Beitrages vom langlebigen Zerfall ist es möglich, auch
eine Darstellung des kurzlebigen Zerfalls zu finden. Sie wird wie folgt charakterisiert.
\begin{align*}
    \lambda_k = (1.61 \pm 0.04) \cdot 10^{-2}\\
    c_k = 6.678 \pm 0.025 \\
    ln(N_{langlebig}) = c_k - \lambda_k \cdot t
\end{align*}

Abbildung \ref{fig:regress} zeigt die jeweiligen Kurven für ihren Geltungsbereich sowie die Kombination aus beiden.

\begin{figure}
 \centering
 \includegraphics[width=\textwidth]{rhodium.pdf}
 \caption{Regressionskurven für den Zerfall von Rhodium.}
 \label{fig:regress}
\end{figure}

\noindent Als Halbwertszeiten ergeben sich aus den Zerfallskonstanten mithilfe der Gleichung \eqref{eqn:t}
\begin{align*}
    T_l = 258.149 \pm 0.001 \si{\s} \\
    T_k = 43.054 \pm 0.025 \si{\s}\\
\end{align*}