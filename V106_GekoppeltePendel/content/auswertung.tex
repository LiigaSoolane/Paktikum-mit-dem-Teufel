\section{Auswertung}
%-------Pendellänge 1-------
\subsection{Pendellänge 1}
Zunächst wurde das Pendel auf eine Länge $l = 0,824 \si{\m}$ eingestellt, um die erste Reihe an Messwerten zu nehmen.

\subsubsection{Frei schwingende Pendel}
Da die Zeit für 5 Schwingungen gestoppt wurde, werden die
Werte erst durch fünf geteilt. Daraus folgen die Messwerte
wie in \ref{tab:frei} dargestellt.
\begin{table}[H]
    \centering
    \caption{Messdaten für frei schwingende Pendel.}
    \label{tab:frei}
    \begin{tabular}{c c}
     \toprule
     $T_1 / \si{\s}$ & $T_2 / \si{\s}$\\
     \midrule
     1.773 & 1.767 \\
     1.792 & 1.777 \\
     1.768 & 1.8 \\
     1.802 & 1.88\\
     1.779 & 1.791\\
     1.837 & 1.894\\
     1.786 & 1.8 \\
     1.804 & 1.817\\
     1.82 & 1.812 \\
     1.795 & 1.829 \\
     \bottomrule
    \end{tabular}
\end{table}
Die Formel für den Mittelwert lautet
\begin{equation}
x = \frac{1}{N} \sum_{k=1}^{N} x_k
\label{eqn:mittel}
\end{equation}
und die für den Fehler des Mittelwertes
\begin{equation}
\sigma_{x} = \frac{\sigma}{\sqrt{N}}
\label{eqn:mif}
\end{equation}
wobei $N$ die Anzahl an Messwerten ist und $\sigma$ die Standardabweichung.
Als Mittelwert aus den Messwerten folgen eine
Schwingungsdauer $T_1 = 1.796\pm 0.007 \si{\s}$ für das eine und 
$T_2 = 1.817\pm 0.012 \si{\s}$ für das andere Pendel.


\subsubsection{Gleich- und gegenphasige Schwingungen}
Als nächstes sollte das System betrachtet werden, wenn die Feder eingesetzt und die Massen tatsächlich verbunden sind.
Durch Teilung errechnen sich die in Tabelle \ref{tab:g1} aufgelisteten Werte. 
\begin{table}
    \centering
    \caption{Messdaten für gleich- ($T_{+}$) und gegenphasige ($T_{-}$).}
    \label{tab:g1}
    \begin{tabular}{c c}
     \toprule
     $T_{+} / \si{\s}$ & $T_{-} / \si{\s}$\\
     \midrule
     1.806 & 1.694 \\
     1.814 & 1.66 \\
     1.782 & 1.65 \\
     1.766 & 1.68 \\
     1.812 & 1.656 \\
     1.838 & 1.674 \\
     1.812 & 1.7\\
     1.818 & 1.676\\
     1.794 & 1.638 \\
     1.818 & 1.662 \\
     \bottomrule
    \end{tabular}
\end{table}
Die Mittelwerte (mithilfe von \eqref{eqn:mittel}, \eqref{eqn:mif}) lauten $T_{+} = 1.806\pm 0.005\si{\s}$ und $T_{-} = 1.669\pm 0.005\si{\s}$. Aus diesen lässt sich der Koppungsgrad $\kappa$ durch die in der 
Theorie hergeleitete Formel berechnen:
\begin{align*}
\kappa = \frac{T_{+}^2 - T_{-}^2}{T_{+}^2 + T_{-}^2}
    = 0.077
\end{align*}
Der zugehörige Fehler wird durch die Gaussche Fehlerfortpflanzung mit den Fehlern des Mittelwertes für $T_{+}$, $T_{-}$ bestimmt.
\begin{equation}
\sigma_f = \sqrt{\sum_{i=1}^{n} \left( \frac{\partial f}{\partial x_i} \right)^2 \sigma_{x_i}^2}
\label{eqn:gauss}
\end{equation}
Für $\kappa$ gilt also:
\begin{align}
\sigma_{\kappa} = \sqrt{\sigma_{T_{+}}^{2} \left(- \frac{2 T_{+} \left(T_{+}^{2} - T_{-}^{2}\right)}{\left(T_{+}^{2} + T_{-}^{2}\right)^{2}} + \frac{2 T_{+}}{T_{+}^{2} + T_{-}^{2}}\right)^{2} + \sigma_{T_{-}}^{2} \left(- \frac{2 T_{-} \left(T_{+}^{2} - T_{-}^{2}\right)}{\left(T_{+}^{2} + T_{-}^{2}\right)^{2}} - \frac{2 T_{-}}{T_{+}^{2} + T_{-}^{2}}\right)^{2}}
\end{align}
Daraus folgt $\kappa = 0.077 \pm 0.005$.

\subsubsection{Gekoppelte Schwingung}
Zudem sollte die dritte mögliche Mode genauer betrachtet werden, die gekoppelte Schwingung. Einerseits ist dabei die Schwingung von Bedeutung, andererseits die
Schwebung, also die Zeit, die das System benötigt, bis dasselbe Pendel wieder in Ruhe ist. \\
Gemessene Daten sind in Tabelle \ref{tab:gek1} zu finden.
\begin{table}
    \centering
    \caption{Messdaten für Schwingungs- ($T$) und Schwebungsdauer ($T_{S}$).}
    \label{tab:gek1}
    \begin{tabular}{c c}
     \toprule
     $T / \si{\s}$ & $T_{S} / \si{\s}$\\
     \midrule
     1.71 & 20.72 \\
     1.664 & 20.85\\
     1.698 & 21.65\\
     1.648 & 21.47\\
     1.692 & 20.72\\
     1.692 & 21.75\\
     1.68 & 20.84\\
     1.67 & 21.0 \\
     1.67 & 20.81\\
     1.67 & 21.18\\
     \bottomrule
    \end{tabular}
\end{table}
Zugehörige Mittelwerte sind mit \eqref{eqn:mittel}, \eqref{eqn:mif} $T = 1.678 \pm 0.006\si{\s}$ und $T_{S} = 21.099 \pm 0.124\si{\s}$. Nun gibt es aber auch eine Formel, mit der sich die Schwebungsdauer theoretisch aus den Schwingungsdauern der
gleich- und gegenphasigen Schwingungen berechnen lassen. Für diesen theoretischen Wert gilt wie zuvor hergeleitet:
\begin{align}
T_\text{S, theoretisch} = \frac{T_{+} \cdot T_{-}}{T_{+} - T_{-}}
\label{eqn:ts}
\end{align}
Da die gleich- und gegenphasigen Moden fehlerbehaftet sind, ist eine Gauss'sche Fehlerrechnung \eqref{eqn:gauss} vonnöten:
\begin{equation}
\sigma_{T_S} = \sqrt{\sigma_{T_{+}}^{2} \left(- \frac{T_{+} T_{-}}{\left(T_{+} - T_{-}\right)^{2}} + \frac{T_{-}}{T_{+} - T_{-}}\right)^{2} + \sigma_{T_{-}}^{2} \left(\frac{T_{+} T_{-}}{\left(T_{+} - T_{-}\right)^{2}} + \frac{T_{+}}{T_{+} - T_{-}}\right)^{2}}
\end{equation}
Also $T_\text{S, theoretisch}= 22.141 \pm 1.444\si{\s}$.

\subsubsection{Frequenzen}
Zum einen lassen sich die Frequenzen $\omega_{+}$, $\omega_{-}$ aus den Messwerten bestimmen mithilfe der bekannten Formel
\begin{equation}
\omega = \frac{2 \pi}{T}
\label{durcht}
\end{equation}
berechnen. Daraus errechnen sich die Messdaten $\omega_{+} = 3.478\pm 0.006\si{\Hz}$ und $\omega_{-} = 3.763\pm 0.007\si{\Hz}$. Für die Schwebungsfrequenz $\omega_{S}=-0.284\pm 0.009\si{\Hz}$ gilt die Formel
\begin{equation}
\omega_{S} = \omega_{+} - \omega_{-}
\end{equation}
Die Fehler folgen mit \eqref{eqn:gauss} aus:
\begin{align}
\sigma_{\omega_{+}} = \sqrt{\frac{\sigma_{T_{+}}^{2}}{T_{+}^{4}}}\\
\sigma_{\omega_{-}} = \sqrt{\frac{\sigma_{T_{-}}^{2}}{T_{-}^{4}}}\\
\sigma_{\omega_{s}} = \sqrt{\sigma_{\omega_{+}}^2 + \sigma_{\omega_{-}}^2}
\end{align}
Zum anderen wurde in der Theorie je eine Formel für die Eigenfrequenzen 
der gleich- und gegenphasigen Schwingungen hergeleitet.
\begin{align}
\omega_\text{+, theoretisch} = \sqrt{\frac{g}{l}}\\
\omega_\text{-, theoretisch} = \sqrt{\frac{g}{l} + \frac{2 \kappa}{l}}\\
\omega_\text{S, theoretisch} = \omega_\text{+, theoretisch} - \omega_\text{-, theoretisch}
\label{eqn:omega}
\end{align}
Für die Länge wird eine Messunsicherheit von $\increment l = 0.005 \si{\m}$ angenommen. Dann gilt für die Fehler nach \eqref{eqn:gauss}:
\begin{align}
\sigma_{\omega_\text{+, theoretisch}} = \sqrt{(\increment l)^2 \left( -\frac{g}{2 \sqrt{\frac{g}{l}}l^2} \right)^2}\\
\sigma_{\omega_\text{-, theoretisch}} = \sqrt{(\increment l)^2 \left( -\frac{2 \kappa + g}{2 \sqrt{\frac{2 \kappa + g}{l}l^2}} \right)^2 + \sigma_{\kappa}^2 \left(\frac{1}{l \sqrt{\frac{2 \kappa + g}{l}}} \right)^2}\\
\sigma_{\omega_\text{S, theoretisch}} = \sqrt{\sigma_{\omega_\text{+, theoretisch}}^2 + \sigma_{\omega_\text{-, theoretisch}}^2}
\label{eqn:omegagauss}
\end{align}
Es sind also: $\omega_\text{+, theoretisch} = 3.449\pm 0.016 \si{\Hz}$, $\omega_\text{-, theoretisch} = \pm 0.056 \si{\Hz}$ und $\omega_\text{S, theoretisch} = -0.027 \pm 0.061\si{\Hz}$.


%-------Pendellänge 2-------


\subsection{Pendellänge 2}
Als zweite Pendellänge wurde das Pendel auf $0.471 \si{\m}$ eingestellt.
\subsubsection{Frei schwingende Pendel}
Auch hier wurden die Daten zunächst durch fünf geteilt, dargestellt
sind sie in Tabelle \ref{tab:frei2}. 
\begin{table}
    \centering
    \caption{Messdaten für frei schwingende Pendel mit geringerer Länge.}
    \label{tab:frei2}
    \begin{tabular}{c c}
     \toprule
     $T_1 / \si{\s}$ & $T_2 / \si{\s}$\\
     \midrule
     1.406 & 1.424 \\
     1.42 & 1.424 \\
     1.406 & 1.412 \\
     1.406 & 1.432\\
     1.408 & 1.406 \\
     1.368 & 1.432 \\
     1.386 & 1.396 \\
     1.394 & 1.424 \\
     1.394 & 1.432 \\
     1.442 & 1.408 \\
     \bottomrule
    \end{tabular}
\end{table}
Als Mittelwerte ergeben sich aus \eqref{eqn:mittel} und \eqref{eqn:mif} $T_1 = 1.403\pm 0.006 \si{\s}$ und $T_2 = 1.419\pm 0.004 \si{\s}$.

\subsubsection{Gleich- und gegenphasige Schwingungen}
Die Feder wurde wieder auf derselben Höhe eingehangen und die ersten beiden Moden wurden betrachtet.
\begin{table}
    \centering
    \caption{Messdaten für gleich- ($T_{+}$) und gegenphasige ($T_{-}$) Mode.}
    \label{tab:g2}
    \begin{tabular}{c c}
     \toprule
     $T_{+} / \si{\s}$ & $T_{-} / \si{\s}$ \\
     \midrule
     1.43 & 1.236 \\
     1.406& 1.272 \\
     1.43 & 1.286 \\
     1.42 & 1.212 \\
     1.402& 1.298 \\
     1.382 & 1.254 \\
     1.456 & 1.262 \\
     1.4240 & 1.27 \\
     1.43 & 1.264 \\
     1.406 & 1.264 \\
     \bottomrule
    \end{tabular}
\end{table}
In Tabelle \ref{tab:g2} sind die Werte aufgeführt, die gemessen wurden. Die Mittelwerte lauten dabei (\eqref{eqn:mittel} und \eqref{eqn:mif}) $T_{+} = 1.419\pm 0.005\si{\s}$ und $T_{-} = 1.262\pm 0.008\si{\s}$. Auch hier sollte wieder der Kopplungsgrad 
der Feder bestimmt werden, mit derselben Formel wie oben ergibt er sich zu $\kappa = 0.117 \pm 0.007$.


\subsubsection{Gekoppelte Schwingung}
Als letzte Mode bleibt wie vorher auch die gekoppelte Schwingung. \ref{tab:gek2} zeigt die Messwerte. \\
$T = 1.264\pm 0.012\si{\s}$ und $T_{S} = 10.513\pm 0.161\si{\s}$ sind die Mittelwerte, wobei wieder \eqref{eqn:mittel} und \eqref{eqn:mif} verwendet wurden und 
$T_{S}$ die Schwebungsdauer ist.
\begin{table}
    \centering
    \caption{Messdaten für Schwingungs- ($T$) und Schwebungsdauer ($T_{S}$).}
    \label{tab:gek2}
    \begin{tabular}{c c}
     \toprule
     $T / \si{\s}$ & $T_{S} / \si{\s}$\\
     \midrule
     1.244 & 10.58\\
     1.264 & 11.13\\
     1.262 & 10.03\\
     1.288 & 11.35\\
     1.256 & 9.84\\
     1.2 & 11.05\\
     1.244 & 10.43\\
     1.344 & 10.1\\
     1.25 & 10.29\\
     1.288 & 10.33\\
     \bottomrule
    \end{tabular}
\end{table}
Durch Berechnung mit der Formel \eqref{eqn:ts} und der Fehlerformel nach Gauss \eqref{eqn:gauss} ergibt sich der Theoriewert $T_\text{S, theoretisch} = 11.416\pm 0.754\si{\s}$.

\subsubsection{Frequenzen}
Zuletzt sollen noch die Frequenzen für diese zweite Pendellänge betrachtet werden.
$\omega_{+}$ und $\omega_{-}$ werden wieder mithilfe der Messwerte und Gleichungen \eqref{eqn:durcht} \eqref{eqn:gauss}
berechnet. Es ergeben sich $\omega_{+} = 4.428\pm 0.011\si{\Hz}$ und $\omega_{-} = 4.98\pm 0.014\si{\Hz}$. Für die Schwebungsfrequenz $\omega_{S}$ ergibt sich 
ein Wert von $-0.54 \pm 0.018\si{\Hz}$.
Die Theoriewerte (siehe \eqref{eqn:omega} und \eqref{eqn:omegagauss}) lauten
\begin{align*}
\omega_\text{+, theoretisch} = 4.563 \pm 0.023\si{\Hz}\\
\omega_\text{-, theoretisch} = 4.617 \pm 0.025\si{\Hz}\\
\omega_\text{S, theoretisch} = -0.054 \pm 0.034\si{\Hz}
\end{align*}